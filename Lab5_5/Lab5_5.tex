% !TeX program = lualatex
% !TeX encoding = utf8
% !TeX spellcheck = uk_UA
% !TeX root =../LabWork.tex


\keywords{Дифракція Френеля,  принцип Гюйгенса-Френеля, зони Френеля, векторна діаграма, інтеграл Френеля}
\abstract{Вивчення законів дифракції від точкового джерела світла на отворі, екрані, щілині та зонній плівці.}
\chapter{Дифракція світла від точкового джерела}
\makeworktitle



\section{Експерементальні подробиці} 
 
    Схема експериментальної установки показана на рисунку (рис.~\ref{fig:FrenselDiffraction}). Когерентний
    паралельний промінь світла від лазера проходить крізь короткофокусну лінзу 2 або систему  лінз.   
    Процес розсіяння світла можна розглядати як утворення уявного точкового джерела випромінювання $S_{0}$. Випромінювання від джерела $S_{0}$ розсвюється на дифракційному елементі 3. Картина дифракції спостерігається на екрані 5. 

    \begin{figure}[!h]
	\centering
	\begin{tikzpicture}[scale=2, every pic/.style={scale=2}, xscale=-1]
%    	\draw (-5,-5) to [grid with coordinates] (5,5);
        \pic at (0,0) {lava};
%        \pic at (2,0.1)  {reuter};
%        \pic (l) at (2,1.8) {lens};
%        \pic (L) at (3,1.8)  {laser};
%        \draw[] (L-top) -- ++(45:1) node[above] {$1$};
%        \pic at (0,0.1) {reuter};
%        \pic (P) at (-0.03,1.8) {difrhole};
%        \draw[] (l-lens) -- ++(-135:1) node[below] {$2$};
%        \draw[] (P-top) -- ++(45:1) node[above] {$3$};
%        \pic at (-3,0.1) {reuter};
%        \pic (E2) at (-3,1.8) {ecran};
%        \draw[] (E2-top) -- ++(45:1) node[above] {$4$};
%        \draw[dash dot] (-4,1.8) -- (4,1.8);
	\end{tikzpicture}
	\caption{Робоча установка}
	\label{fig:FrenselDiffraction}	
    \end{figure} 

\section{Хід експерименту}

\subsection{Визначення радіуса отвору}

\begin{enumerate}
\item Вводьте  в розширений пучок світла транспаранти з різними розмірами отворів.

 Пересуваючи уздовж оптичної лави транспарант $3$, поспостерігайте на екрані $4$ характерні дифракційні картини, які змінюють одна одну. 

\item   Фотографуйнте дифракційну картину. Фіксуючи положення ($a$, $b$) транспаранту, коли чітко спостерігається темна та світла плями. Перевірте формулу~\eqref{eq:Zone_Diameter}.
\end{enumerate} 

\subsection{Спостереження плями Пуассона}

\begin{enumerate}
\item Вводьте  в розширений пучок світла транспаранти з різними розмірами непрозорих круглих дисків. Зверніть увагу на те, що в центрі дифракційної картини спостерігається світна пляма (пляма Пуассона).
\item Сфотографуйте дифракційну картину. Занотуйте результати експерименту.
\end{enumerate}
\subsection{Фокусавання світла за допомогою зонної платівки}
 \begin{enumerate}
\item Вводьте  в розширений пучок світла зонну платівку Френеля. Пересуваючи на лаві екран $4$ зафіксуйте положення, яке відповідає фокусуючій дії платівки. Знайдіть максимальну фокусну відтань платівки та радіус першої зони платівки. Перевірте формулу~\eqref{eq:Focuse}.
\item Сфотографуйте дифракційну картину. Занотуйте результати експерименту. 
\end{enumerate}  

\section*{Контрольні запитання}
\addcontentsline{toc}{section}{Контрольні запитання}
\begin{enumerate}[label*=\arabic*.]
	\item Що таке дифракція. На які типи вона поділяється? Який критерій для цього використовується?
	\item Сформулюйте принцип Френеля.
	\item Запишіть інтеграл Френеля.
	\item Поясніть принцип дії зонної платівки Френеля.
    \item Виведіть формули~\eqref{eq:Fr-KirchSpher_m}, \eqref{eq:Zone_Diameter}, \eqref{eq:Focuse}.
\end{enumerate} 

