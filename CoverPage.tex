% !TeX program = lualatex
% !TeX encoding = utf8
% !TeX spellcheck = uk_UA
% !TeX root =../EMProblems.tex

%========================================================================================================
%
%									      Палітурка
%
%========================================================================================================

% ---------------------------------------- Кольори секцій -----------------------------------------------
\definecolor{themecolordark}{RGB}{34,102,101}
\definecolor{themecolorlight}{RGB}{153,168,167}
\definecolor{titlebgdark}{RGB}{0,103,102}
\definecolor{titlebglight}{RGB}{191,233,251}

\newcommand{\CoverPage}{
	\begin{alwayssingle}
		\begin{center}
			\begin{flushright}\bfseries\sffamily
				\MakeUppercase{Міністерство освіти і науки України}\\
				КПІ ім. Ігоря Сікорського\\
			\end{flushright}
			\begin{tcolorbox}[titlepagestyle,
					toprule=0.10cm,
					bottomrule=0.10cm,
					overlay={%
						\node (picture) at ([xshift=4cm]frame.west) {\includegraphics{logo_PTI}};
					}
			]%
			\begin{flushright}
				\large\bfseries\color{white}Фізико-технічний інститут
			\end{flushright}
			\end{tcolorbox}

			\vspace*{15em}

			\begin{tcolorbox}[
				titlepagestyle,
				toprule=0.15cm,
				bottomrule=0.15cm,
				top=1.3cm,
				bottom=0.7cm,
				overlay={%
				\node[%
							fill=white,
							rounded corners = 15pt,	
							draw=themecolorlight,
							line width=0.15cm,
							inner sep=0pt,
							text width=17cm,
							minimum height=2cm,
							align=center,
							%anchor=east,
							font=\sffamily\bfseries\large
						] (title) at (frame.north) {С.~М. Пономаренко,
													В.~В. Іванова%
					};
				}
			]
			\centering
			\Huge\sffamily\bfseries\textcolor{white}{\realtitle}\\
			\huge\sffamily\bfseries\textcolor{white}{\subtitle}
			\end{tcolorbox}	
			\vfill
%		 	\begin{Large}\color{themecolordark!90!black}
%			\begin{gather*}
%				\frac{\partial F^{\mu\nu}}{\partial x^\nu} =                                                                                -\frac{4\pi}{c}j^\mu \\
%				\frac{\partial F^{\mu\nu}}{\partial x^\alpha} + \frac{\partial F^{\mu\alpha}}{\partial x^\nu} +\frac{\partial F^{\alpha\mu}}{\partial x^\nu}  = 0.
%			\end{gather*}
%			\end{Large}
			\vfill
			\begin{tcolorbox}[titlepagestyle,
					toprule=0.10cm,
					bottomrule=0.10cm]
				\begin{center}\color{white}\bfseries\normalsize
					\MakeUppercase{Київ~2021} \\
%					КПІ ім. Ігоря Сікорського \\
%					\the\year
				\end{center}			       	
			\end{tcolorbox}
		\end{center}
		\clearpage
	\end{alwayssingle}
\setcounter{page}{1}	
}


%========================================================================================================
%
%									      Титульна сторінка
%
%========================================================================================================

\renewcommand\maketitle{
	\begin{alwayssingle}
		\begin{center}
				\MakeUppercase{Міністерство освіти і науки України}
				
				\bigskip
				\MakeUppercase{Національний технічний університет України}\\
				<<КИЇВСЬКИЙ ПОЛІТЕХНІЧНИЙ ІНСТИТУТ \\ імені ІГОРЯ СІКОРСЬКОГО>>
				\vspace*{150pt}
		
%				{\large С.~Пономаренко,
%						Ю.~О.~Тараненко
%						}
%				\vspace*{50pt}
			
				{\Huge\sffamily\bfseries\realtitle}\\[1em]
				{\huge\sffamily\bfseries\subtitle}	
			
			\vspace*{50pt}
			\begin{center}\itshape
                Рекомендовано Методичною радою КПІ ім. Ігоря Сікорського як навчальний посібник для здобувачів ступеня бакалавра за освітньою програмою <<Прикладна фізика>> спеціальності 105 Прикладна фізика та наноматеріали 
%				Методичні вказівки до виконання лабораторних робіт за темою <<Змінний струм>> з навчальної дисципліни <<Електрика та магнетизм>> для студентів, які навчаються за спеціальністю 105 <<Прикладна фізика та наноматеріали>>
			\end{center}

			\vfill
			\begin{center}
				\MakeUppercase{Київ} \\
				КПІ ім. Ігоря Сікорського \\
				2021
			\end{center}			       	
		\end{center}
		\clearpage
	\end{alwayssingle}	
}


%========================================================================================================
%
%									      Друга сторінка
%
%========================================================================================================
\newcommand\makeinfopage{
	\begin{alwayssingle}
		\noindent%	
        \begin{minipage}[t]{\textwidth}
                \realtitle: \subtitle\ [Електронний ресурс] : навч. посіб. для студ. спеціальностей
                105 <<Прикладна фізика та наноматеріали>> / КПІ ім. Ігоря Сікорського; уклад.  С.~М. Пономаренко, В.~В. Іванова ; КПІ ім. Ігоря Сікорського.~--- Електронні текстові дані
            (1 файл: 0.7~МБ). – Київ : КПІ ім. Ігоря Сікорського, 2021. --- \the\numexpr\getpagerefnumber{LastPage}~с.
        \end{minipage}
        \vspace*{2em}
		\begin{center}\itshape\small
				Гриф надано Методичною радою КПІ ім. Ігоря Сікорського (протокол № 6 від 25.02.2021 р.) за поданням Вченої ради Фізико-технічного інституту (протокол №1 від 11.01.2021 р.)
		\end{center}
		\begin{center}
			Електронне мережне навчальне видання
			%\par {Версія від~\href{http://www.istpravda.com.ua/dates}{\today}} \par\else \par  \fi
		\end{center}
		\begin{center}\bfseries
    		\LARGE\sffamily\realtitle\\
   			\Large\sffamily\subtitle
   		\end{center}

		\vfill\noindent%
        \begin{minipage}[t]{0.2\linewidth}
            	\begin{flushleft}
                    Укладачі:
                \end{flushleft}
        \end{minipage}\hfill
        \begin{minipage}[t]{0.78\linewidth}
		\begin{flushleft}
			\href{http://phes.ipt.kpi.ua/ponomarenko-sergij-mikolajovich}{\itshape Пономаренко Сергій Миколайович}, к.ф.-м.н., доцент \\
			\itshape \href{http://apd.ipt.kpi.ua/kafedra/perspages/IvanovaVV.html}{Іванова Віта Вікторівна}, к.т.н., доцент
		\end{flushleft}
        \end{minipage}

        \vspace*{2em}
		\noindent%
        \begin{minipage}[t]{0.2\linewidth}
            	\begin{flushleft}
                    Відповідальний редактор:
                \end{flushleft}
        \end{minipage}\hfill
        \begin{minipage}[t]{0.78\linewidth}
                к.ф.-м.н., доцент \href{http://ipt.kpi.ua/smirnov}{Смирнов С.~О.}
        \end{minipage}

        \vspace*{2em}
		\noindent%
        \begin{minipage}[t]{0.2\linewidth}
            	\begin{flushleft}
                    Рецензент:
                \end{flushleft}
        \end{minipage}\hfill
        \begin{minipage}[t]{0.78\linewidth}
                \href{http://www.nas.gov.ua/UA/PersonalSite/Pages/default.aspx?PersonID=0000007584}{Лимаренко Руслан Анатолійович}, к.ф.-м.н., с.н.с., учений секретар Міжнародного центру <<Інститут прикладної оптики>>  НАН України
        \end{minipage}

        \vfill

		Розглянуто зміст, основні складові та порядок виконання лабораторних робіт з дисципліни <<Оптика>>. 

		Для студентів фізико-технічного інституту КПІ ім. Ігоря Сікорського, які навчаються за спеціальністю 105~<<Прикладна фізика та наноматеріали>>.
		
		\vfill
				
%		\begin{flushleft}\small
%			Ілюстративний матеріал підручника підготовлений за допомогою пакету \href{http://pgf.sourceforge.net}{TikZ/Pgf}. Верстка тексту проведена в видавничій системі \LaTeXe{} (компілятор Lua\LaTeX) на базі системи комп'ютерної верстки \TeX{} (Збірка  \href{https://www.tug.org/texlive/}{\TeX Live~\the\year}) з використанням оболонки \href{https://www.texstudio.org}{\TeX Studio}.
%		\end{flushleft}	
	\begin{flushright}
        \textcopyright{} С.~М. Пономаренко, В.~В. Іванова, 2021 р.
    \end{flushright}
		\newpage%
	\end{alwayssingle}
}



