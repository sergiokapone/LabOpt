% !TeX program = lualatex
% !TeX encoding = utf8
% !TeX spellcheck = uk_UA
% !TeX root =../LabWork.tex


\keywords{Поляризація світла,  закон Малюса, кут Брюстера, формули Френеля}
\abstract{Дослідити експериментально ефекти, пов’язані з поляризацією світла. Перевірити закон Малюса і закон Брюстера.}
\chapter{Поляризація світла}
\makeworktitle



%\section{Експерементальні подробиці} 
 
%    Схема експериментальної установки показана на рисунку (рис.~\ref{fig:FrenselDiffraction}). Когерентний
%    паралельний промінь світла від лазера проходить крізь короткофокусну лінзу 2 або систему  лінз.   
%    Процес розсіяння світла можна розглядати як утворення уявного точкового джерела випромінювання $S_{0}$. Випромінювання від джерела $S_{0}$ розсвюється на дифракційному елементі 3. Картина дифракції спостерігається на екрані 5. 
%
%    \begin{figure}[!h]
%	\centering
%	\begin{tikzpicture}[scale=2, every pic/.style={scale=2}]
%%    	\draw (-5,-5) to [grid with coordinates] (5,5);
%        \pic at (0,0) {lava};
%        \pic at (2,0.1)  {reuter};
%        \pic (l) at (2,1.8) {lens};
%        \pic (L) at (3,1.8)  {laser};
%        \draw[] (L-top) -- ++(45:1) node[above] {$1$};
%        \pic at (0,0.1) {reuter};
%        \pic (P) at (0.05,1.8) {difrhole};
%        \draw[] (l-lens) -- ++(-135:1) node[below] {$2$};
%        \draw[] (P-top) -- ++(45:1) node[above] {$3$};
%        \pic at (-3,0.1) {reuter};
%        \pic (E2) at (-3,1.8) {ecran};
%        \draw[] (E2-top) -- ++(45:1) node[above] {$4$};
%        \draw[dash dot] (-4,1.8) -- (4,1.8);
%	\end{tikzpicture}
%	\caption{Робоча установка}
%	\label{fig:FrenselDiffraction}	
%    \end{figure} 

%Для реальних поляризаторів вводиться параметр $V$, що має назву ефективності
%поляризатора. Ця величина визначається як відношення інтенсивності компоненти світла,
%поляризованого у дозволеному напрямку, до інтенсивності компоненти з ортогональним
%до дозволеного напрямком поляризації на виході з поляризатору при падінні на останній неполяризованого світла. Параметр $V$ характеризує якість поляризаційного пристрою.

\section{Експериментальні подробиці}

Дозволений напрям обох поляроїдів не співпадає з нулем відліку кута і визначається у експерименті з кутом Брюстера. Множник при переключенні шкали міліамперметру може відрізнятись від 10 і потребувати експериментального визначення.
При встановленні чорного дзеркала під кутом Брюстера впевніться, що кут падіння
дорівнює куту відбивання (наближено --- за шкалою, точно --- за максимумом інтенсивності відбитого світла).

Зверніть увагу на те, що фотодатчик є дуже чутливим до фонового освітлення. Рекомендується
звести до мінімуму вплив світла зовнішніх джерел.

\section{Хід експерименту}

\subsection{Перевірка закону Малюса}
\begin{enumerate}
\item Змонтуйте на оптичній лаві освітлювач із лампою розжарення. Впевніться, що пучок світла після лінзи є близьким до паралельного. У разі необхідності відрегулюйте положення лінзи. Встановіть на лаві фотодатчик, під’єднайте міліамперметр.
\item Зафіксуйте, на скільки зменшується інтенсивність світла, що потрапляє у фотодатчик, якщо між освітлювачем та фотодатчиком встановити поляризатор. Перевірте, чи за-лежить інтенсивність світла від кута встановлення поляризатору.
\item \label{item:MaluseLaw}Встановіть другий поляризатор і зніміть залежність інтенсивності світла від кута повороту одного з поляризаторів (при фіксованому куті другого).
\end{enumerate}

\subsection{Дослідження світла, відбитого від чорного Дзеркала}
\begin{enumerate}
\item \label{item:Surface}Спрямуйте світло від освітлювача на чорне дзеркало (поліровану діелектричну платівку) через поляризатор. Поверніть чорне дзеркало під кутом $30\ldots60^\circ$ до напряму розповсюдження світла. Обертаючи поляризатор, добийтеся мінімальної інтенсивності світла, що відбивається від дзеркала. Таким чином, використовуючи дзеркало в якості аналізатора, знайдіть дозволені напрями поляризатора.
\item \label{item:FrenselFormulae} Встановіть поляризатор так, щоб його дозволений напрям лежав у площині падіння для чорного дзеркала (тобто горизонтально). Обертаючи чорне дзеркало і вимірюючи інтенсивність світла, що відбивається від нього, знайдіть кут падіння, при якому ця інтенсивність буде мінімальною (кут Брюстера). Зафіксуйте цей кут.
\item \label{item:MaluseMirror} Встановіть чорне дзеркало під кутом Брюстера. За допомогою приладу для вимірювання фотоструму виміряйте залежність інтенсивності світла, що відбивається від чорного дзеркала, від кута повороту поляризатора (аналогічно до п.~\ref{item:FrenselFormulae} експерименту, але в якості аналізатора виступає чорне дзеркало).
\end{enumerate}

\subsection{Дослідження світла, що відбивається і проходить крізь стопу Столетова}
\begin{enumerate}
\item Встановіть поляризатор так, як в п.~\ref{item:Surface}. Обертаючи стопу і вимірюючи інтенсивність світла, що відбивається від неї та проходить, знайдіть кут Брюстера.
\item Встановіть стопу під кутом Брюстера. За допомогою приладу для вимірювання фотоструму виміряйте залежність інтенсивності світла, що пройшло крізь стопу, переконайтесь, що воно лінійно поляризоване. Зніміть залежність інтенсивності від кута повороту поляризатора (аналогічно до п.~\ref{item:FrenselFormulae} експерименту, але в якості аналізатора виступає стопа).
\end{enumerate}

\section{Обробка результатів експерименту}

\begin{enumerate}
\item Користуючись даними, отриманими в пункті \ref{item:MaluseLaw} лабораторної роботи, побудуйте графік залежності інтенсивності світла, що пройшло через систему двох поляризаторів, від квадрату косинуса кута між їх дозволеними напрямами. (Оскільки поляризатори не є ідеальними, при апроксимації використовуйте варіант МНК для прямої з вільним членом.) Перевірте вірність закону Малюса. Визначте з графіку ефективність поляроїду.
\item Користуючись даними, отриманими в пункті ~\ref{item:FrenselFormulae}  лабораторної роботи, побудуйте графік залежності інтенсивності світла, що відбилось від чорного дзеркала, від кута повороту дзеркала. Апроксимуйте її формулою~\eqref{eq:FrenselParallel} і визначте показник заломлення $n$. Порівняйте отриманий результат з табличним значенням $n$.
\item Побудуйте графік залежності інтенсивності від кута повороту для системи <<поляроїд --- чорне дзеркало>> (п.~\ref{item:MaluseMirror} експерименту). Перевірте вірність закону Малюса. Визначте з графіку ефективність чорного дзеркала як поляроїда.
\end{enumerate}

\section*{Контрольні запитання}
\addcontentsline{toc}{section}{Контрольні запитання}
\begin{enumerate}[label*=\arabic*.]
\item Яке світло називають лінійно поляризованим? Поляризованим по колу?
\item Як розрахувати інтенсивність світла, що пройшло через поляризатор, аналізатор?
\item Які типи поляризаторів ви знаєте?
\item Яку величину називають ступенем поляризації?
\item Як змінюється ступінь поляризації відбитого світла при зміні кута падіння від $0$ до $90^\circ$?
\item Чим поляризатор відрізняється від аналізатора?
\item Що являє собою стопа Столєтова і для чого вона служить?
\item Як отримати світло, поляризоване по колу? Що таке пів- та чвертьхвильова платівка?
\item Що таке дихроїзм? Наведіть  приклади  двопроменезаломлюючих кристалів.
\item Чи можна виготовити поляризатор з металу?
\item У чому полягає явище обертання напрямку поляризації?
\item Які речовини називають оптично активними?
\item Як влаштований цукрометр?
\end{enumerate}
