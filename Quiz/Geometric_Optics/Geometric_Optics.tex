%%============================ Compiler Directives =======================%%
%%                                                                        %%
% !TeX program = pdflatex							    	
% !TeX encoding = utf8
% !TeX spellcheck = uk_UA
%%                                                                        %%
%%============================== Клас документа ==========================%%
%%                                                                        %%
\documentclass[12pt]{article}
%%                                                                        %%
%%============================= Мови та кодування ========================%%
%%                                                                        %%
\usepackage[utf8]{inputenc}
\usepackage[T2A,T1]{fontenc}
\usepackage[english, russian, ukrainian]{babel}
%%                                                                        %%
%%=========================== Киририличні корекції =======================%%   
\usepackage{indentfirst}
\usepackage{cmap}
\IfFileExists{ukrcorr.sty}{\usepackage{ukrcorr}}{}
%%                                                                        %%
%%============================= Геометрія сторінки =======================%%
%%                                                                        %%        
\usepackage[%
	a4paper,%
	footskip=1cm,%
	headsep=0.3cm,% 
	top=2cm, %поле сверху
	bottom=2cm, %поле снизу
	left=2cm, %поле ліворуч
	right=2cm, %поле праворуч
    ]{geometry}
%%                                                                        %%
%%================================= Шрифти ===============================%%   
%%                                                                        %%
%\fontsize{12pt}{13pt}\selectfont                                         %%
%%============================== Інтерліньяж  ============================%%
%%                                                                        %%
\renewcommand{\baselinestretch}{1}
%-------------------------  Подавление висячих строк  --------------------%%
\clubpenalty =10000
\widowpenalty=10000
%---------------------------------Інтервали-------------------------------%%
\setlength{\parskip}{0.5ex}%
\setlength{\parindent}{2.5em}%
%%                                                                        %%
%%                                                                        %%
%%=========================== Математичні пакети і графіка ===============%%
%%                                                                        %%
\usepackage{amsmath}
\usepackage{graphicx}
\usepackage{enumitem}
\usepackage{xcolor}


%%                                                                        %%
%%================================ Інші пакети ===========================%%   
%%                                                                        %%
%%========================== Гіперпосилення (href) =======================%%
%%                                                                        %% 
\usepackage[%colorlinks=true,
	%urlcolor = blue, %Colour for external hyperlinks
	%linkcolor  = malina, %Colour of internal links
	%citecolor  = green, %Colour of citations
	bookmarks = true,
	bookmarksnumbered=true,
	unicode,
	linktoc = all,
	hypertexnames=false,
	pdftoolbar=false,
	pdfpagelayout=TwoPageRight,
	pdfauthor={Ponomarenko S.M. aka sergiokapone},
	pdfdisplaydoctitle=true,
	pdfencoding=auto
	]%
	{hyperref}
		\makeatletter
	\AtBeginDocument{
	\hypersetup{
		pdfinfo={
		Title={\@title},
		}
	}
	}
	\makeatother
%%                                                                        %%	
%%============================== Оформлення списків=======================%%
%%                                                                        %%
\usepackage{enumitem}
\setlist{nolistsep, leftmargin=0cm,itemindent=.5cm}
%%                                                                        %%
%%============================ Заголовок та автори =======================%%
%%                                                                        %%
\title{Питання до захисту лабораторної роботи за темою\\ {\bfseries Геометрична оптика та оптичні прилади}}
\author{}
\date{}                                   
%%                                                                        %%
%%========================================================================%%


\begin{document}
\maketitle

\noindent%
\begin{enumerate}[label*=\bfseries\color{red!60!black}\arabic*.,ref=\arabic*, wide,leftmargin=*]
\item Перерахуйте найбільш поширені джерела світла і классифицируйте їх. У чому полягають відмінності роботи люмінесцентних ламп від ламп розжарювання? Які переваги та недоліки люмінесцентних ламп в порівнянні з лампами розжарювання?
\item Що таке <<точкове джерело світла>>? Як отримати таке джерело світла?
\item Визначте поняття <<світловий промінь>>. В яких випадках світлові промені прямолінійні? Криволінійні?
\item Які бувають світлові пучки?
\item Які межі застосування геометричної оптики?
\item У чому полягає закон прямолінійного поширення світла? Наведіть приклади, в яких проявляється цей закон, і приклади, де спостерігаються відступи від нього?
\item Як утворюються тінь і півтінь? Як отримати від предмета тільки півтінь? Різку тінь без півтіні? В яких випадках утворюються нерізкі тіні?
\item Що називають відносним показником заломлення середовища? Як він взаємопов'язаний зі швидкостями поширення світла в середовищах? З абсолютними показниками заломлення середовища?
\item Як класифікуються середовища за оптичними густинами? Як змінюється швидкість світла в середовищі зі збільшенням її оптичної густини?
\item Чому дорівнює швидкість світла у вакуумі?
\item Якими методами вимірюють швидкість світла? У чому труднощі вимірювання швидкості світла?
\item Визначте поняття «світловий потік», «сила світла», «освітленість». Як освітленість залежить від сили світла джерела? Від світлового потоку, що падає на освітлену поверхню?
\item Перерахуйте найбільш поширені приймачі світлової енергії.
\item Яке призначення і принцип дії фотометрів?
\item Сформуйте основні закони геометричної оптики, принцип оборотності світлових променів. 
\item Сформулюйте принцип Гюйгенса. Виведіть на його основі закони відбивання та заломлення.
\item Що таке геометричний та оптичний шлях променя?
\item Сформулюйте принцип Ферма. Виведіть на його основі закони відбивання та заломлення.
\item Чим відрізняється дифузне відбиття світла від дзеркального? Матова поверхня від дзеркально відображає?
\item Які зображення називають дійсними? Уявними? Чи реальне уявне зображення? Чи можливо уявне зображення отримати на екрані?
\item Як побудувати зображення точкового предмету в плоскому дзеркалі? Предмета, розміри якого більше розмірів дзеркала?
\item Як відрізнити свою фотографію від фотографії свого зображення в плоскому дзеркалі?
\item Як класифікують сферичні дзеркала? Визначте основні елементи, що характеризують сферичні дзеркала: фокус, оптичні осі, оптичний центр, головне фокусна відстань і т.д.
\item Як розрахувати оптичну силу сферичного дзеркала? Чи відрізняється оптична сила опуклого дзеркала від оптичної сили увігнутого дзеркала, якщо радіуси їх сферичних поверхонь чисельно рівні? Чому дорівнює оптична сила плоского дзеркала?
\item Які промені зазвичай використовують при побудові зображення в сферичних дзеркалах?
\item Чому фокуси одних дзеркал називають дійсними, а інших-уявними?
\item Як побудувати зображення точкового предмету, що знаходиться на головній оптичній осі (зміщеного з неї) увігнутого (опуклого) сферичного дзеркала, при його різному видаленні від дзеркала?
\item Якщо плисти на човні по спокійній поверхні озера і спостерігати його дно, то здається, що найглибше місце весь час знаходиться якраз під човном. Чому?
\item Чому після заходу сонце темніє не відразу, а настають сутінки?
\item В чому полягає явище повного внутрішнього відбивання світла? За яких умов воно можливе? Який кут називається граничним кутом повного внутрішнього відображення?
\item Чи завжди кут заломлення більше кута падіння?
\item Як використовується повне відбивання світла при оберненні світлових пучків? 
\item Джерело світла і спостерігач перебувають під водою. За яких умов це джерело спостерігачеві здасться розташованим під водою?
\item Як залежить величина відбитого світлового потоку від кута падіння? Від показників заломлення середовищ, на межі яких відбувається відображення світу?
\item Побудуйте хід світлового променя, що падає на плоскопараллельну пластинку нормально (під деяким кутом), якщо пластинка поміщена в однорідне середовище, а показник заломлення речовини пластинки більше (менше) показника середовища. Те ж для двох пластинок з різними показниками заломлення.
\item Побудуйте хід променя в тригранній призмі так, щоб промінь: а) відхилявся до її основи; б) відхилявся до її заломлюючого кута; в) зазнавав на одній її граней повне внутрішнє відбиття.
\item Дайте визначення понять «лінза», «тонка лінза». Перерахуйте відомі типи лінз і їх відмінні риси.
\item Визначте основні елементи, що характеризують лінзи: фокуси, фокальні площині, оптичні осі тощо.
\item Які лінзи називають збиральними (розсіювальними)? Як слід змінити властивості навколишнього середовища, щоб збирає лінза стала розсіювальною?
\item Виведіть формулу тонкої лінзи (розгляньте різні типи лінз і різні місця розташування предмета відносно лінзи). Вкажіть обмеження, при яких справедлива формула тонкої лінзи. Викладіть правило знаків при застосуванні формули тонкої лінзи.
\item За якими формулами визначається оптична сила тонких лінз різних типів? Чому оптична сила тонкої лінзи залежить від властивостей навколишнього середовища?
\item Чому фокуси одних лінз називають дійсними, а інших уявними? Які предмети (зображення) називають дійсними і які уявними?
\item Які промені зазвичай застосовуються при побудові зображень у тонких лінзах? Чим відрізняється побудова зображень в розсіювальній лінзі в порівнянні зі збиральною лінзою?
\item Як змінюється місце розташування і збільшення зображення предмета в збиральній (розсіювальній) тонкій лінзи при переміщенні предмета уздовж головної оптичної осі цієї лінзи з нескінченності до її оптичного центру?
\item Побудуйте зображення а) точкового предмету, що знаходиться на головній оптичній осі (зміщеного з неї) збиральної (розсіювальної) тонкої лінзи, при його різному видаленні від оптичного центру цієї лінзи, б) протяжного предмета, розміри якого більше розмірів лінзи.
\item За якими формулами визначається лінійне збільшення лінзи?
\item Побудуйте хід променів (зображення предметів) в найпростіших оптичних приладах (мікроскоп, телескоп, фотоапарат).
\item Як визначити силу світла джерела, що є зображенням точкового джерела світла в дзеркалі (плоскому, опуклому і увігнутому сферичному)?
\item Який принцип роботи ока як оптичної системи?
\item Око дає дійсне, зменшене і обернене зображення предметів. Чому все довкола нам не здаються перевернутими?
\item Чим відрізняються оптичні сили лінз, що виправляють короткозорий очей, від оптичних сил лінз, що виправляють далекозоре око?
\item Чому дорівнює оптична сила двох тонких лінз складених разом?
\item Чому дорівнює оптична сила двох тонких лінз, якщо вони розділені середовищем з показником заломлення $n$, а відстань між лінзами $d$?
\item Чому дорівнює оптична сила товстої лінзи?
\item Якщо ми хочемо використовувати збиральну лінзу як лупу, то предмет повинен розташовуватися до лінзи ближче або далі, ніж її фокус? Чому?
\item Як визначається збільшення оптичного приладу (мікроскопу, телескопу)?
\item Чому дорівнює відстань найкращого зору?
\item Чи можна отримати збільшене зображення в розсіювальній лінзі?
\item Чому телескопи рефрактори роблять довгими?
\item Де розташовані головні площини товстої лінзи?
\end{enumerate}
\end{document}
