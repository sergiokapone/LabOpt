% !TeX program = lualatex
% !TeX encoding = utf8
% !TeX spellcheck = uk_UA
% !BIB program = biber
\documentclass{Syllabus}
\addbibresource{d:/Projects/LaTeX/MyPackage/MyBase.bib}


% ================================================
%                Дані  дисципліни
% ================================================
\logokaf{logokaf.png}
\creator{доцент, к.т.н., доцент Іванова Віта Вікторівна} % Розробник програми
\creator{доцент, к.ф.-м.н., доцент Пономаренко Сергій Миколайович} % Розробник програми
\lecturer[vivanova950@gmail.com ]{доцент, к.т.н., доцент Іванова Віта Вікторівна}
\practicer[vivanova950@gmail.com ]{доцент, к.т.н., доцент Іванова Віта Вікторівна}
%\laborer[s.ponomarenko@kpi.ua]{доцент, к.ф.-м.н., доцент Пономаренко Сергій Миколайович}
\laborer[]{ст. викл. Бех Станіслав Вікторович}
\disciplinename{Оптика} % Назва дисципліни
\kursurl{https://do.ipo.kpi.ua/enrol/index.php?id=1988}
\shedule{http://ipt.kpi.ua/navchalnij-protses}
\kafname{Прикладної фізики}
% -------------- Номера протоколів ---------------
\methodcomday{27} % Число ухвалення метод момісією
\methodcommonth{червня} % місяць ухвалення метод момісією
\methodcomyear{2024}
\methodcomnum{6} % Номер протокола ухвалення метод момісією
\kafday{11} % Число ухвалення кафедрою
\kafmonth{червня} % місяць ухвалення кафедрою
\kafyear{2024}
\kafnum{{6}}% Номер протокола ухвалення кафедрою
% ------------------------------------------------
\efield{10 Природничі науки}
\teachingprogramm{Прикладна фізика}
\speciality{105 Прикладна фізика та наноматеріали}
\level{Перший (бакалаврський)} % освітньо-кваліфікаційний рівень
\status{Нормативна}
%-------- Семестр 3 --------------------
\course{2}
\semestr{весняний}
\controltype{екзамен}
\AudDistribHours%
{7.5}%
{36}%
{36}%
{72}%

\MKR{1}
\DKR{}
\RGR{1}
% ================================================


\def\zsrs{\textit{Завдання на СРС:}}
\def\lit{\textit{Література для опрацювання:\ }}
\def\probl{\textit{Задачники:\ }}
\def\lab{\textit{Лабораторний практикум:\ }}


\begin{document}
\printhead


\syllabuschapter{Реквізити навчальної дисципліни}
\bluetableprint

\syllabuschapter{Програма навчальної дисципліни}

\section{Опис навчальної дисципліни, її мета, предмет вивчання та результати навчання}

%Викладач обґрунтовує необхідність вивчення навчальної дисципліни, відповідаючи на питання «Чому майбутньому фахівцю варто вчити саме цю дисципліну?», визначає мету, предмет дисципліни та програмні результати 3 навчання (компетентності, знання, уміння, навички, досвід, послідовність дій в стандартних виробничих ситуаціях тощо), які студент/аспірант набуде після вивчення дисципліни з розподілом на окремі освітні компоненти (якщо дисципліна вивчається декілька семестрів).


%Всі ми звикли до того, що, використовуючи електричну мережу, можна легко включити освітлення, зарядити комп'ютер, запустити нагрівач або кондиціонер. Всі ці прилади вимагають для своєї роботи енергію, яка доставляється по електричних мережах, що з'єднує генератори, розташовані на електричних станціях зі споживачами. Але ж електричні мережі з'явилися за історичними мірками зовсім недавно --- всього лише сто років тому. Про північне сяйво знають всі в силу грандіозності і пишності цього явища. Але не всі знають, що північне сяйво виникає внаслідок прориву заряджених частинок, які безперервно випускає Сонце, через магнітне поле Землі. Якби магнітне поле Землі захищала її поверхню від сонячного вітру (так образно називається випускається Сонцем потік заряджених частинок), життя на Землі було б неможливе. У нашому курсі ми розглянемо основні явища і закони, пов'язані з магнітним полем. Перебуваючи в автомобілі, ви звично включаєте радіоприймач і GPS-навігатор. Робота всіх цих приладів, що припускає зв'язок з віддаленими контрагентами, заснована на розповсюдженні радіохвиль. У радіохвилі електричне та магнітне поле тісно пов'язані один з одним, що дає їм можливість поширюватися з величезною швидкістю. Те ж саме відбувається в світлових хвилях, за рахунок яких ми бачимо навколишній світ, тільки електричне та магнітне поля змінюються в світловій хвилі швидше, ніж в радіохвилі. Один з предметів нашого розгляду --- природа електромагнітних хвиль (що включають радіохвилі, світлові хвилі і хвилі інших діапазонів) і умови їх випромінювання. Чи можна в реальності здійснити зависання будь-якого предмета в повітрі? Так, можна, якщо використовувати надпровідні матеріали. Надпровідність --- явище, яке спостерігається при низьких температурах в деяких речовинах, в яких електричний струм може текти без опору. Надпровідники мають незвичайні магнітні властивості, що і дозволяє здійснити так звану <<левітацію>>.

Курс <<\discipline>> є частиною курсу загальної фізики. У ньому вивчаються оптичні явища, тобто явища, які ров'язані із закономірностями випромінювання, поширення світла та його взаємодією з речовиною.

Мета курсу <<\discipline>> полягає в тому, щоб довести до студента основні принципи вчення про оптичні явища в логічній послідовності теорії цих явищ і процесів та зв'язку з іншими розділами фізики, докладно пояснюючи фізичні закони в практичному застосуванні. Завданнями вивчення даної дисципліни є досягнення розуміння фізичної суті оптичних явищ, що дає можливість сформувати фізичну картину світу, а також розуміти сучасні технології, що базуються на цих явищах та використовувати їх для розробки наукоємних технологій.

%Сама дисципліна <<\discipline>> освітлює три кола питань. До першого кола належать основні поняття і загальні принципи, що керують електричними і магнітними явищами; до другої --- електричні і магнітні властивості речовини; до третьої --- технічні і практичні застосування вчення про електромагнітні явища. Найбільша увага в цьому курсі приділяється питанням, що першої групи, які викладаються з найбільшою повнотою. Прийнято індуктивний метод викладу. Основні поняття і принципи встановлюються шляхом узагальнення дослідних фактів. Процес узагальнення йде безперервно і цілеспрямовано протягом усього викладу, завершуючи в середині курсу встановленням системи рівнянь Максвелла.

Після засвоєння навчальної дисципліни студенти мають продемонструвати такі результати навчання:

\begin{enumerate}
    \item[\bfseries знання:] природи світла та шкали електромагнітних хвиль; геометричної оптики і теорії центрованих оптичних систем; хвильової оптики і кристалооптики; квантової оптики і властивості фотонів; основ Фур'є-оптики; лінійну фізичну оптику (ефекти Фарадея, Керра, Зеемана);
    \item[\bfseries уміння:] користуватися законами геометричної оптики та теорією центрованих оптичних систем для побудови хода променів й визначення характеристик нескладних оптичних приладів; розраховувати інтерференційну або дифракційну картини при освітленні простіших систем когерентним монохроматичним світлом; розраховувати роздільну здатність, нормальне збільшення і нормальну ширину щілини у оптичних приладах;
    \item[\bfseries досвід:] вільно орієнтуватися на якісному й кількісному рівні в основних фізичних явищах, пов'язаних з проявами оптичними явищами, виробити навички практичного використання засвоєних знань, методів і підходів у подальшому навчанні та професійній діяльності.
\end{enumerate}

Згідно з вимогами освітньо-наукової програми
%\footnote{\href{https://docs.google.com/document/d/1IOGn2p1Boq6-4fbsm9HKYVe9Bki8GPg-/edit}{ОСВІТНЬО-НАУКОВА ПРОГРАМА
%<<Прикладна фізика>> спеціальності 105 Прикладна фізика та наноматеріали другого (магістерського) рівня вищої освіти}}
студенти після засвоєння навчальної дисципліни <<\discipline>> мають продемонструвати такі результати навчання:

\subsection*{Загальні компетентності СВО}

\begin{enumerate}
\item [ЗК 1:] Здатність застосовувати знання у практичних ситуаціях.
\item [ЗК 2:] Знання та розуміння предметної області та розуміння професійної діяльності.
%\item [ЗК 3:] Здатність спілкуватися державною мовою як усно, так і письмово.
%\item [ЗК 4:] Здатність спілкуватися іноземною мовою.
%\item [ЗК 5:] Здатність використання інформаційних і комунікаційних технологій.
\item [ЗК 6:] Здатність проведення досліджень на відповідному рівні.
%\item [ЗК 7:] Здатність до пошуку, оброблення та аналізу інформації з різних джерел.
%\item [ЗК 8:] Здатність застосовувати навички міжособистісної взаємодії.
%\item [ЗК 9:] Здатність працювати автономно.
%\item [ЗК 10:] Здатність здійснювати безпечну діяльність.
%\item [ЗК 11:] Здатність реалізувати свої права і обов’язки як члена суспільства, усвідомлювати цінності громадянського (вільного демократичного) суспільства та необхідність його сталого розвитку, верховенства права, прав і свобод людини і громадянина в Україні.
%\item [ЗК 12:] Здатність зберігати та примножувати моральні, культурні, наукові цінності і досягнення суспільства на основі розуміння історії та закономірностей розвитку предметної області, її місця у загальній системі знань про природу і суспільство та  розвитку суспільства, техніки і технологій, використовувати різні види та форми рухової активності для активного відпочинку та ведення здорового способу життя.
%\item [ЗК 13:] Здатність критично оцінювати результати своєї діяльності в професійній сфері, навчанні і приймати обґрунтовані рішення з урахуванням наукових, соціальних, етичних, правових, економічних аспектів.
%\item [ЗК 14:] Здатність продовжувати навчання із значним ступенем самостійності
\end{enumerate}

\subsection*{Фахові компетентності СВО}

\begin{enumerate}
%\item [ФК 1:] Здатність брати участь у плануванні та виконанні наукових та науково-технічних проектів.
\item [ФК 2:] Здатність брати участь у плануванні і виконанні експериментів та лабораторних досліджень властивостей фізичних систем, фізичних явищ і процесів, обробленні й презентації їхніх результатів.
\item [ФК 3:] Здатність брати участь у виготовленні експериментальних зразків, інших об'єктів дослідження.
%\item [ФК 4:] Здатність брати участь у впровадженні результатів досліджень та розробок.
%\item [ФК 5:] Здатність до постійного розвитку компетентностей у сфері прикладної фізики, інженерії та комп’ютерних технологій.
%\item [ФК 6:] Здатність використовувати сучасні теоретичні уявлення в галузі фізики для аналізу фізичних систем.
%\item [ФК 7:] Здатність використовувати методи і засоби теоретичного дослідження та математичного моделювання в професійній діяльності.
%\item [ФК 8:] Здатність працювати в колективах виконавців, у тому числі в міждисциплінарних проектах.
%\item [ФК 9:] Здатність брати участь у роботах з проведення наукових досліджень властивостей явищ і процесів у фізичній та/або біофізичній, енергетичній системі, та зі складання наукових звітів з виконаних робіт.
%\item [ФК 10:] Здатність розуміти та застосовувати апарат спеціальних розділів математики для розв’язання проблем прикладної фізики, моделювати фізичні процеси і системи, використовуючи статистичні та стохастичні методи, комп’ютерну графіку, та представляти результати моделювання.
%\item [ФК 11:] Здатність використовувати знання основ професійно-орієнтованих дисциплін для виконання наукових досліджень, розв’язання практичних проблем прикладної фізики та для самостійного  опанування нових технологій, в тому числі із суміжних галузей, застосовувати отримані знання і практичні навички для прийняття інноваційних рішень при розв’язанні складних практичних задач або в навчанні, зокрема,  високих фізичних технологій та/або фізики живих систем та/або фізики енергетичних систем.
\end{enumerate}

\subsection*{Програмні результати навчання}

\begin{enumerate}
\item [ПРН 1:] Знати і розуміти сучасну фізику на рівні, достатньому для розв’язання складних спеціалізованих задач і практичних проблем прикладної фізики.
%\item [ПРН 2:] Застосовувати сучасні математичні методи для побудови й аналізу математичних моделей фізичних процесів.
\item [ПРН 3:] Застосовувати ефективні технології, інструменти та методи експериментального дослідження властивостей речовин і матеріалів, включаючи наноматеріали, при розв’язанні практичних проблем прикладної фізики.
\item [ПРН 4:] Застосовувати фізичні, математичні та комп'ютерні моделі для дослідження фізичних явищ, розробки приладів і наукоємних технологій.
%\item [ПРН 5:] Вибирати ефективні методи та інструментальні засоби проведення досліджень у галузі прикладної фізики.
%\item [ПРН 6:] Відшуковувати необхідну науково-технічну інформацію в науковій літературі, електронних базах, інших джерелах, оцінювати надійність та релевантність інформації.
%\item [ПРН 7:] Класифікувати, аналізувати та інтерпретувати науково-технічну інформацію в галузі прикладної фізики.
%\item [ПРН 8:] Вільно спілкуватися з професійних питань державною та англійською мовами усно та письмово.
%\item [ПРН 9:] Презентувати результати досліджень і розробок фахівцям і нефахівцям, аргументувати власну позицію.
%\item [ПРН 10:] Планувати й організовувати результативну професійну діяльність індивідуально і як член команди при розробці та реалізації наукових і прикладних проєктів.
%\item [ПРН 11:] Знати цілі сталого розвитку та можливості своєї професійної сфери для їх досягнення, в тому числі в Україні.
%\item [ПРН 12:] Розуміти закономірності розвитку прикладної фізики, її місце в розвитку техніки, технологій і суспільства, у тому числі в розв'язанні екологічних проблем.
%\item [ПРН 13:] Оцінювати фінансові, матеріальні та інші витрати, пов’язані з реалізацією проектів у сфері прикладної фізики, соціальні, екологічні та інші потенційні наслідки реалізації проектів.
\item [ПРН 14:] Обирати та використовувати методи та засоби дослідження структури, складу та речовин і матеріалів.
%\item [ПРН 15:] Знання основ методології наукових досліджень в прикладній фізиці, технологіії  оформлення, презентації та захисту результатів наукових досліджень, вміння складати звіти з виконаних робіт.
%\item [ПРН 16:] Знання методів аналізу випадкових процесів, теорії ймовірності і математичної статистики, програмування, комп’ютерної графіки, прикладних програм і методів обчислень, методів розв’язання рівнянь математичної фізики, теорії функції комплексної змінної, тензорного аналізу, для розуміння сучасних фізичних теорій і розв’язання проблем прикладної фізики та моделювання процесів, що відбуваються в фізико-технічних системах.
%\item [ПРН 17:] Знання основ професійно-орієнтованих дисциплін спеціальності, зокрема  хімії, ядерної фізики, статистичної радіофізики та оптики, електродинаміки суцільних середовищ для розв’язання практичних проблем прикладної фізики, в т.ч. високих фізичних технологій та/або фізики живих систем та/або фізики енергетичних систем.
\end{enumerate}


\section{Пререквізити та постреквізити дисципліни (місце в структурно-логічній схемі\\ навчання за відповідною освітньою програмою)}

%Зазначається перелік дисциплін, або знань та умінь, володіння якими необхідні студенту (вимоги до рівня підготовки) для успішного засвоєння дисципліни (наприклад, «базовий рівень володіння англійською мовою не нижче А2»). Вказується перелік дисциплін які базуються на результатах навчання з даної дисципліни.

Для засвоєння матеріалу курсу <<\discipline>> студенти повинні знати курс фізики в рамках шкільної програми та засвоїти термінологію та поняття курсів:
\begin{enumerate}
    \item Математичний аналіз;
    \item Тензорний аналіз;
    \item Механіка;
    \item Термодинаміка та молекулярна фізика;
    \item Електрика та магнетизм.
\end{enumerate}

Також повинні вміти використовувати математичний апарат: операції з матрицями,  диференціювати, інтегрувати.

Отримані практичні навички та засвоєні теоретичні знання під час вивчення навчальної дисципліни <<\discipline>> можна використовувати в подальшому в навчальних дисциплінах, пов’язаних з теоретичними та практичними аспектами прикладної фізики, зокрема:

\begin{enumerate}
    \item Класична механіка;
    \item Теорія поля;
    \item Електродинаміка суцільних середовищ;
    \item Статистична оптика та радіофізика.
\end{enumerate}


\section{Зміст навчальної дисципліни}

%Надається перелік розділів і тем всієї дисципліни.

\begin{Rozdil}
\item Геометрична оптика.
    \begin{Rozdil}
        \item Базові поняття.
        \item Геометрична оптика.
        \item Фотометрія.
    \end{Rozdil}
\item Хвильова оптика.
    \begin{Rozdil}
        \item Поширення електромагнітних хвиль у однорідних середовищах.
        \item Інтерференція.
        \item Дифракція.
        \item Основи Фур’є-оптики.
    \end{Rozdil}
\item Фізична оптика.
    \begin{Rozdil}
        \item Оптичні явища в кристалах.
        \item Фізична оптика.
        \item Розсіювання та поглинання світла.
        \item Теплове випромінювання.
    \end{Rozdil}
\end{Rozdil}

%Зазначається: базова (підручники, навчальні посібники) та додаткова (монографії, статті, документи, електронні ресурси) література, яку потрібно прочитати або використовувати для опанування дисципліни.
%
%Можна надати рекомендації та роз’яснення:
%\begin{itemize}
%\item де можна знайти зазначені матеріали (бібліотека, методичний кабінет, інтернет тощо);
%\item що з цього є обов’язковим для прочитання, а що факультативним;
%\item як саме студент/аспірант має використовувати ці матеріали (читати повністю, ознайомитись тощо);
%\end{itemize}
% зв’язок цих ресурсів з конкретними темами дисципліни.
%Бажано зазначати не більше п’яти базових джерел, які є вільно доступними, та не більше 20
%додаткових.


	\section{Навчальні матеріали та ресурси}\label{sec:refsec}

    Нижче наводиться перелік навчальних матеріалів та ресурсів для засвоєння матеріалу, розглядуваного на лекційних заняттях та для додаткового вивчення.

\nocite{%
    Smytyna,
    Romaniuk,
    BornWolf,
    Siv4,
    LandsbergOptics,
    Godzhaev,
    AleshkevichOptics,
    Butikov,
    Mat4,
    Ling,
    Ivanova,
    Ovchinkin2,
    Kozel,
    SivP4,
    FTILabPractOptics,
    FTILabPractOpticsNew,
}

\printbibliography[category=Main, heading=subbibliography, title={Основна}]
\addtocategory{Main}{%
    Smytyna,
    Romaniuk,
}

\printbibliography[category=Additional, heading=subbibliography, title={Додаткова}]
\addtocategory{Additional}{%
    BornWolf,
    Siv4,
    LandsbergOptics,
    Godzhaev,
    AleshkevichOptics,
    Butikov,
    Mat4,
    Ling,
}


\printbibliography[category=Problems, heading=subbibliography, title={Задачники}]
\addtocategory{Problems}{%
    Ivanova,
    Ovchinkin2,
    Kozel,
    SivP4,
}

\printbibliography[category=LabPract, heading=subbibliography, title={Лабораторний практикум}]
\addtocategory{LabPract}{%
    FTILabPractOptics,
    FTILabPractOpticsNew,
}


\syllabuschapter{Навчальний контент}

\section{Методика опанування навчальної дисципліни (освітнього компонента)}

%Надається інформація (за розділами, темами) про всі навчальні заняття (лекції, практичні,
%семінарські, лабораторні) та надаються рекомендації щодо їх засвоєння (наприклад, у формі
%календарного плану чи деталізованого опису кожного заняття та запланованої роботи).
\vspace*{3ex}
\subsection*{Лекційні заняття}
\vspace*{3ex}
\begin{longtable}{|>{\arraybackslash}m{0.03\linewidth}|>{\raggedright\arraybackslash}m{0.9\linewidth}|}
\hline
 \thead{№} & \thead {Назва теми лекції та перелік основних питань} \\
\hline
\endhead
\multicolumn{2}{|c|}{Розділ 1. Геометрична оптика.} \\\hline % ---------------------------
\multicolumn{2}{|c|}{Тема 1.1. Базові поняття.} \\\hline % ---------------------------
\thead{\rownumber.}
& \textbf{Базові поняття}.
Еволюція парадигми оптики і уявлень про природу світла. Геометрична, хвильова, квантова і нелінійна оптика. Електромагнітна природа світла. Шкала електромагнітних хвиль. Оптичні явища, що доводять хвильовий характер світла. Досліди, які підтверджують квантові властивості світла. Фотоефект. Ефект Комптона. Фотони --- кванти світла, їх енергія, імпульс та момент імпульсу. Принцип роботи лазера. Загальне уявлення про випромінювання Вавілова-Черенкова і люмінесценцію.
\\\hline
\multicolumn{2}{|c|}{Тема 1.2. Геометрична оптика.} \\\hline % ---------------------------
\thead{\rownumber.}
& \textbf{Основні положення}.
Принцип Ферма. Закони заломлення та відбивання світла. Параксіальне наближення. Формула тонкої лінзи. Побудова зображення у лінзах та сферичних дзеркалах.
\\\hline
\thead{\rownumber.}
& \textbf{Центровані оптичні системи}. v
Центровані оптичні системи. Кардинальні точки. Кутове, поперечне та поздовжнє збільшення. Діафрагми, апертурні кути і кут поля зору. Уявлення про аберації оптичних систем.
\\\hline
\thead{\rownumber.}
& \textbf{Основні оптичні прилади.}
Телескоп, мікроскоп, фотоапарат, тощо.
\\\hline
\multicolumn{2}{|c|}{Тема 1.3. Фотометрія.} \\\hline % ---------------------------
\thead{\rownumber.}
& Основні поняття фотометрії: світловий потік, сила світла, освітленість, яскравість та їх одиниці. Освітленість та яскравість зображення.
\\\hline
\multicolumn{2}{|c|}{Розділ 2. Хвильова оптика.} \\*\hline % ---------------------------
\multicolumn{2}{|c|}{Тема 2.1. Поширення електромагнітних хвиль у однорідних середовищах.} \\*\hline % ------
\thead{\rownumber.}
& \textbf{Поширення електромагнітних хвиль у однорідних середовищах}.
Наближення скалярних хвиль. Плоскі і сферичні хвилі. Монохроматичні хвилі, їх частота і хвильовий вектор. Векторні електромагнітні хвилі, їх поперечність. Поляризація світла. Плоскополяризоване світло. Кругова та еліптична поляризація. Методи отримання і аналізу поляризованого світла. Закон Малюса. Природне світло.
Заломлення та відбивання світла на границі двох однорідних середовищ. Формули Френеля. Коефіцієнти проходження та відбивання світла. Поляризація при заломленні та відбиванні. Кут Брюстера.
\\*\hline
\multicolumn{2}{|c|}{Тема 2.2. Інтерференція.} \\*\hline % ---------------------------
\thead{\rownumber.}
& \textbf{Інтерференція при розділенні фронту}.
Двопроменева інтерференція. Просторова та часова когерентність світла. Вплив когерентності на видність інтерференційних смуг. Зв’язок між тривалістю цугу та шириною спектра. Лазери як джерело когерентного світла. Інтерферометр Релея. Зоряний інтерферометр Майкельсона.
\\\hline
\thead{\rownumber.}
& \textbf{Інтерференція при розділенні амплітуди}.
Інтерференція при відбиванні світла. Смуги рівного нахилу і рівної товщини, їх локалізація.
Багатопроменева інтерференція. Пластина Люммера-Герке. Інтерферометр Фабрі-Перо.
\\\hline
\multicolumn{2}{|c|}{Тема 2.3. Дифракція.} \\*\hline % ---------------------------
\thead{\rownumber.}
& \textbf{Дифракція Френеля та Фраунгофера}.
Принцип Гюйгенса-Френеля. Зони Френеля. Дифракція Френеля та Фраунгофера. Дифракція Фраунгофера на щілині та круглому отворі. Дифракція Френеля на щілині і краю екрана. Спіраль Корню.
\\\hline
\thead{\rownumber.}
& \textbf{Дифракційна ґратка}.
Дисперсія, дисперсійна область і роздільна здатність дифракційної грат¬ки та інших спектральних приладів. Критерій роздільності Релея. Границі застосування геометричної оптики. Роздільна здатність мікроскопа і телескопа.
\\\hline
\multicolumn{2}{|c|}{Тема 2.4. Основи Фур'є оптики.} \\*\hline % ---------------------------
\thead{\rownumber.}
& Хвильове поле як суперпозиція плоских хвиль. Уявлення про звичайну та об’ємну голографію
\\\hline
\multicolumn{2}{|c|}{Розділ 3. Фізична оптика.} \\*\hline % ---------------------------
\multicolumn{2}{|c|}{Тема 3.1. Оптичні явища у кристалах.} \\*\hline % ---------------------------
\thead{\rownumber.}
& \textbf{Закони поширення світла в однорідному анізотропному середовищі}. Рівняння Френеля кристалооптики. Променева поверхня та поверхня хвильових нормалей. Еліпсоїд Френеля та оптична індикатриса. Одновісні та двовісні кристали. Подвійне променезаломлення. Штучна анізотропія. Явище Керра. Двовісні кристали --- явище конічної рефракції. Поширення світла в поглинаючому анізотропному середовищі. Плеохроїзм, лінійний та круговий дихроїзм.
\vspace*{6ex}
\\\hline
\multicolumn{2}{|c|}{Тема 3.2. Фізична оптика.} \\*\hline % ---------------------------
\thead{\rownumber.}
& \textbf{Дисперсія світла}. Нормальна та аномальна дисперсія світла. Хроматична аберація. Фазова та групова швидкість хвиль. Класична теорія дисперсії. Формула Лорентца-Лоренца. Діелектрична проникність плазми. Відбивання радіохвиль від іоносфери.
\textbf{Вплив магнітного поля на світло}. Явище Фарадея. Ефект Зеемана.
\textbf{Дифракція рентгенівських променів}. Умови Брегга-Вульфа. Загальне уявлення про рентгеноструктурний аналіз.
\\\hline
\multicolumn{2}{|c|}{Тема 3.3. Розсіювання та поглинання світла.} \\*\hline % ---------------------------
\thead{\rownumber.}
& Закон Бугера-Ламберта-Бера. Релеївське розсіювання. Поляризація світла, що було розсіяне. Залежність інтенсивності розсіювання від частоти. Природні явища, зумовлені релеївським та нерелеївським розсіюванням світла.
\\\hline
\multicolumn{2}{|c|}{Тема 3.4. Теплове випромінювання.} \\*\hline % ---------------------------
\thead{\rownumber.}
& Закони Стефана-Больцмана, Віна, Релєя. Розподіл Планка.
\\\hline
\end{longtable}

\subsection*{Практичні заняття}

Необхідний матеріал, для підготовки до практичних занять можна знайти, зокрема, у~\cite{Ivanova}, який містить основні формули, необхідні для розв'язування задач. В кінці збірника міститься довідковий матеріал та перелік літератури для підготовки. Також є сторінка практичної частини в системі Moodle за адресою~\url{https://do.ipo.kpi.ua/course/view.php?id=1936} та в кампусі за адресаим:
\begin{enumerate}[label=$\bullet$]
    \item \url{https://campus.kpi.ua/tutor/index.php?mode=mob\&show\&irid=179065}
    \item \url{https://campus.kpi.ua/tutor/index.php?mode=mob\&show\&irid=182362}
    \item \url{https://campus.kpi.ua/tutor/index.php?mode=mob\&show\&irid=183006}
    \item \url{https://campus.kpi.ua/tutor/index.php?mode=mob\&show\&irid=187064}
    \item \url{https://campus.kpi.ua/tutor/index.php?mode=mob\&show\&irid=187065}
    \item \url{https://campus.kpi.ua/tutor/index.php?mode=mob\&show\&irid=188644}
    \item \url{https://campus.kpi.ua/tutor/index.php?mode=mob\&show\&irid=189422}
    \item \url{https://campus.kpi.ua/tutor/index.php?mode=mob\&show\&irid=191245}
    \item \url{https://campus.kpi.ua/tutor/index.php?mode=mob\&show\&irid=192115}
\end{enumerate}


\begin{center}\setcounter{magicrownumbers}{0}
        \begin{longtable}{|>{\arraybackslash}m{0.03\linewidth}|>{\raggedright\arraybackslash}m{0.9\linewidth}|}
        \hline
         \thead{№} & \thead {Назва теми заняття та перелік розглядуваних питань}
        \\\hline
        \endhead
		\rownumber. & Принцип Ферма. Закони заломлення та відбивання світла. Повне внутрішнє відбивання.
        \newline \probl{} \cite{Ivanova}
        \\\hline
		\rownumber. & Побудова зображення у тонких лінзах та сферичних дзеркалах
        \newline \probl{}  \cite{Ivanova}
		\\\hline
		\rownumber. & Центровані оптичні системи. Кардинальні точки, їх знаходження. Розрахунок кутового, поперечного та поздовжнього збільшення.
        \newline \probl{}  \cite{Ivanova}
		\\\hline
		\rownumber. & Діафрагми, апертурні кути і кут поля зору.
        \newline \probl{}  \cite{Ivanova}
		\\\hline
		\rownumber. & Основні поняття фотометрії. Розрахунок світлового потоку, сили світла, освітленості, яскравості.
        \newline \probl{}  \cite{Ivanova}
		\\\hline
		\rownumber. & Освітленість та яскравість зображення.
        \newline \probl{}  \cite{Ivanova}
		\\\hline
		\rownumber. & Плоскі і сферичні скалярні хвилі. Монохроматичні хвилі, їх частота і хвильовий вектор. Поляризація світла. Закон Малюса.
        \newline \probl{}  \cite{Ivanova}
		\\\hline
		\rownumber. & Коефіцієнти проходження та відбивання світла на границі двох однорідних середовищ. Поляризація при заломленні та відбиванні. Кут Брюстера.
        \newline \probl{}  \cite{Ivanova}
		\\\hline
		\rownumber. & Двопроменева інтерференція.
        \newline \probl{}  \cite{Ivanova}
		\\\hline
		\rownumber. & Інтерференція при відбиванні світла. Смуги рівного нахилу і рівної товщини. Кільця Ньютона.
        \newline \probl{}  \cite{Ivanova}
		\\\hline
		\rownumber. &  Зони Френеля. Робота з діаграмою Френеля.
        \newline \probl{}  \cite{Ivanova}
		\\\hline
		\rownumber. & Дифракція Фраунгофера на щілині та круг¬ло¬му отворі. Дифракційна ґратка.
        \newline \probl{}  \cite{Ivanova}
    	\\\hline
		\rownumber. & Дисперсія, дисперсійна область і роздільна здатність дифракційної грат¬ки та інших спектральних приладів. Роздільна здатність мікроскопа і телескопа.
        \newline \probl{}  \cite{Ivanova}
		\\\hline
		\rownumber. & Подвійне променезаломлення. Платівка у чверть довжини хвилі.
        \newline \probl{}  \cite{Ivanova}
		\\\hline
		\rownumber. & Ефект Керра. Явище Фарадея. Ефект Зеемана.
        \newline \probl{}  \cite{Ivanova}
        \\\hline
		\rownumber. & Закон Бугера-Ламберта-Бера. Релеївське розсіювання. Закони Стефана-Больцмана, Віна, Релєя. Розподіл Планка.
        \newline \probl{}   \cite{Ivanova}
		\\\hline
		\rownumber. & Голографія. Екскурсія-семінар до Міжнародного центру <<Інститут прикладної оптики>> НАН України
        \newline \probl{}  \cite{Ivanova}
		\\\hline
		\rownumber. & Голографія. Екскурсія-семінар до Міжнародного центру <<Інститут прикладної оптики>> НАН України
        \newline \probl{}  \cite{Ivanova}
		\\\hline
	\end{longtable}
\end{center}

\vspace*{-3em}
\subsection*{Лабораторні заняття}\setcounter{magicrownumbers}{0}

\begin{longtable}{|>{\arraybackslash}m{0.03\linewidth}|>{\raggedright\arraybackslash}m{0.9\linewidth}|}
\hline
 \thead{№} & \thead {Назва теми заняття} \\
\hline
\endhead
\thead{\rownumber.}
& Вимірювання фокусних відстаней лінз.
\newline \lab{}\cite{FTILabPractOptics}
\\\hline
\thead{\rownumber.}
& Оптичні прилади.
\newline \lab{}\cite{FTILabPractOptics}
\\\hline
\thead{\rownumber.}
& Інтерференція світла.
\newline \lab{}\cite{FTILabPractOpticsNew}
\\\hline
\thead{\rownumber.}
& Кільця Ньютона.
\newline \lab{}\cite{FTILabPractOpticsNew}
\\\hline
\thead{\rownumber.}
& Дифракція світла від точкового джерела.
\newline \lab{}\cite{FTILabPractOpticsNew}
\\\hline
\thead{\rownumber.}
& Дифракція в паралельних променях та принцип невизначеності.
\newline \lab{}\cite{FTILabPractOpticsNew}
\\\hline
\thead{\rownumber.}
& Вивчення поляризованого світла.
\newline \lab{}\cite{FTILabPractOpticsNew}
\\\hline
\thead{\rownumber.}
& Дослідження спектру ртутної лампи за допомогою гоніометра та ртутної лампи.
\newline \lab{}\cite{FTILabPractOptics}
\\\hline
\thead{\rownumber.}
& Дослідження дисперсії призми за допомогою гоніометра та ртутної лампи.
\newline \lab{}\cite{FTILabPractOptics}
\\\hline
\end{longtable}

\section{Самостійна робота студента}

Самостійна робота студентів має на меті розвиток творчих здібностей та активізація їх розумової діяльності, формування потреби безперервного самостійного поповнення знань та розвиток морально-вольових зусиль. Завданням самостійної роботи студентів є навчити студентів самостійно працювати з літературою, творчо сприймати навчальний матеріал і осмислювати його та формування навичок до щоденної роботи з метою одержання та узагальнення знань, умінь і навичок.


На самостійну роботу відводяться наступні види завдань:
\begin{enumerate}[label=$\bullet$]
\item обробка і осмислення інформації, отриманої безпосередньо на заняттях;
\item робота з відповідними підручниками та особистим конспектом лекцій;
%\item самостійне вивчення окремих тем або питань із розробкою конспекту;
%\item робота з відповідною літературою;
\item виконання підготовчої роботи до лабораторних, практичних занять та до написання МКР;
\item виконання РГР;
\item підготовка до складання семестрового контролю.
\end{enumerate}

%Наприкінці семестру студенти за бажанням готують питання за вибором у вигляді невеликого реферату або дослідження (теоретичного або експериментального). Теми питань обирають самостійно, але за узгодженням з викладачем. Студент має проявити творчій підхід, вміння працювати з літературою, самостійно поставити проблему, передбачити суміжні питання, що тут виникають.
%



\syllabuschapter{Політика та контроль}

\section{Політика навчальної дисципліни (освітнього компонента)}

%Зазначається система вимог, які викладач ставить перед студентом/аспірантом:
%\begin{itemize}\setlength\itemsep{0ex}
%\item правила відвідування занять (як лекцій, так і практичних/лабораторних);
%\item правила поведінки на заняттях (активність, підготовка коротких доповідей чи текстів,
%відключення телефонів, використання засобів зв’язку для пошуку інформації на гугл-
%диску викладача чи в інтернеті тощо);
%\item правила захисту лабораторних робіт;
%\item правила захисту індивідуальних завдань;
%\item правила призначення заохочувальних та штрафних балів;
%\item  політика дедлайнів та перескладань;
%\item політика щодо академічної доброчесності;
%\item  інші вимоги, що не суперечать законодавству України та нормативним документам
%Університету.
%\end{itemize}

\subsection*{Відвідування занять}
Відвідування лекцій, а також відсутність на них, не оцінюється. Однак, студентам рекомендується відвідувати заняття, оскільки на них викладається теоретичний матеріал та розвиваються навички, необхідні для успішного складання \control у. В разі великої кількості пропусків студент може бути недопущений до \control у.

\subsection*{Пропущені контрольні заходи}

Результат модульної контрольної роботи для студента, який не з’явився на контрольний захід, є нульовим. У такому разі, студент має можливість написати модульну контрольну роботу, але максимальний бал за неї буде дорівнювати 50~\% від загальної кількості балів. Повторне написання модульної контрольної роботи не допускається.

\subsection*{Календарний рубіжний контроль}

Проміжна атестація студентів (далі --- атестація) є календарним рубіжним контролем. Метою проведення атестації є підвищення якості навчання студентів та моніторинг виконання графіка освітнього процесу студентами\footnote{Рейтингові системи оцінювання результатів навчання: Рекомендації до розроблення і застосування. Київ: КПІ ім. Ігоря Сікорського, 2018. 20 с.}.

\begin{center}
\begin{tabular}{|l|c|c|}
\hline
Термін атестації                            & \thead{Перша атестація\\ 8-й тиждень}     & \thead{Друга атестація\\ 14-й тиждень}     \\\hline
\thead{Критерій: поточний контроль}         & $\ge 20$~ балів                           & $\ge 30$~ балів \\\hline
\end{tabular}
\end{center}

\subsection*{Академічна доброчесність}

Політика та принципи академічної доброчесності визначені у розділі 3 Кодексу честі Національного технічного університету України «Київський політехнічний інститут імені Ігоря Сікорського». Детальніше: \url{https://kpi.ua/code}.

\subsection*{Норми етичної поведінки}

Норми етичної поведінки студентів і працівників визначені у розділі 2 Кодексу честі Національного технічного університету України «Київський політехнічний інститут імені Ігоря Сікорського». Детальніше: \url{https://kpi.ua/code}.

\subsection*{Процедура оскарження результатів контрольних заходів}

Студенти мають можливість підняти будь-яке питання, яке стосується процедури контрольних заходів та очікувати, що воно буде розглянуто згідно із наперед визначеними процедурами (згідно <<Положення про систему забезпечення якості вищої освіти у Національному технічному університеті України «Київський політехнічний інститут імені Ігоря Сікорського>>, <<Положення про організацію навчального процесу>>).

\section{Види контролю та рейтингова система оцінювання результатів навчання (РСО)}

%Вказуються всі види контролю та бали за кожен елемент контролю, наприклад:
%Поточний контроль: експрес-опитування, опитування за темою заняття, МКР, тест тощо
%Календарний контроль: провадиться двічі на семестр як моніторинг поточного стану
%виконання вимог силабусу.
%
%Семестровий контроль: екзамен / залік / захист курсового проекту (роботи)
%Умови допуску до семестрового контролю: мінімально позитивна оцінка за індивідуальне
%завдання / зарахування усіх лабораторних робіт / семестровий рейтинг більше ХХ балів.

Видами контролю успішності засвоєння матеріалу дисципліни є  модульна контрольна робота (МКР), розрахунково-графічна робота (РГР) та семестровий контроль.


\subsection*{Активність на практичних заняттях}

\pgfmathsetmacro{\pracBal}{10}
На практичних заняттях за кожну самостійно розв’язану біля дошки задачу дається до $1$ бал. Конструктивна ідея або вірна відповідь з <<місця>>: $0.5$ балів. Можливі і інші варіанти оцінки роботи на розсуд викладача, що веде практику, проте прикінцевий максимальний бал становить не більше~$\pracBal$.

З огляду на обмежену кількість виходів до дошки студенти зацікавлені у активній участі в роботі на практичних заняттях.

\subsection*{Лабораторні роботи}

\pgfmathsetmacro{\labBal}{20}
За кожну вчасно здану лабораторну роботу студент отримує дві оцінки за експериментальну та теоретичну частину (наприклад 5/4) згідно із п'яти-бальною шкалою, або 100 бальною шкалою (на розсуд викладача, який веде лабораторні заняття). Перед кожною із атестацій та наприкінці семестру виводиться середня оцінка. Рейтингові бали нараховуються згідно із наступною схемою: середня оцінка за національною шкалою $\times 4$. Можливі і інші варіанти оцінки на розсуд викладача, що веде лабораторні роботи, проте прикінцевий максимальний бал становить не більше~$\labBal$.



\subsection*{Модульна контрольна робота}

\pgfmathsetmacro{\mkrBal}{10}

Модульна контрольна робота проводиться після завершення першої частини курсу <<\discipline>> проводиться протягом 2-х академічних годин на практичних заняттях. Вона складається з $4$ задач і передбачає письмовий розв’язок задачі, подібних до тих, що розглядались на практичних заняттях та під час виконання домашніх робіт.

Оцінюється за чіткими критеріями з позначенням коректної або некоректної відповіді, а також з коментарями, зауваженнями тощо. Критерії оцінювання модульної контрольної роботи:
\begin{enumerate}[label=$\bullet$]
\item максимальна кількість балів за кожне питання – повна правильна відповідь, 95\% інформації, там де треба наведено рисунки, позначення, є письмовий коментар щодо базових понять та законів, які використовуються під час розв’язку задачі,
\item 75\% балів --- розв’язок правильний, не всі умови попереднього пункту виконано,
\item 60\% балів --- наведено основні базові поняття для розв’язку, розв’язок неправильний.
\item списані відповіді, які студент не може пояснити, не зараховуються.
\end{enumerate}


\subsection*{Розрахунково-графічна робота}

\pgfmathsetmacro{\rgrBal}{20}

РГР складається із двох частин і містить задачі, що задаються студентам для самостійної роботи після завершення кожної теми. Задачі оформлюються в окремому зошиті (або їх розв'язки оформлюються в електронному вигляді за допомогою \LaTeX{}) послідовно за заданими темами і мають містити: умову, рисунок там, де необхідно, пояснення до формул, чіткі позначення, чисельні обрахунки, відповідь із розмірністю отриманої величини.

РГР приймається у два етапи. Захист першої частини РГР --- проводиться протягом 2-х академічних годин на консультаціїї після завершення першого розділу курсу <<\discipline>>, другої частини --- на останньому тижні на основному занятті і консультаціях. За кожний етап захисту дається максимум $10$ балів. Студенту надається можливість захищати кожну половину розрахункової роботи до трьох разів. За кожну невдалу спробу призначаються штрафні задачі, кількість яких визначається на розсуд викладача, що проводить практику (загальна кількість не більше $5$). У разі, якщо три спроби невдалі --- студент, отримавши $10$ штрафних балів, наприкінці семестру здає всю роботу повністю.


Можливі і інші варіанти оцінки на розсуд викладача, що веде практику, проте загальна кількість балів за РГР не більше \rgrBal.

Критерії оцінювання за один етап. На захисті студент повинен:
\begin{enumerate}[label=$\bullet$]
\item показати зошит з не менш як 75\% оформлених задач~ --- до 2 балів
здати для перевірки оформлену розрахункову роботу з необхідним мінімумом задач (90\%), що визначається згідно із тематикою РГР. У разі необхідності виправити неправильні розв’язки~--- до 5 балів
\item прокоментувати розв’язок довільно вибраних викладачем задач із розрахункової роботи (кількість задач визначається викладачем, але не менше 5).
\item  Розв’язати в присутності викладача штрафні задачі, кількість яких визначається на розсуд викладача, що проводить практику~--- до 10 балів.
\end{enumerate}


\subsection*{Умови допуску до \control у}

В таблиці наведені умови допуску до семестрового контролю.

\begin{center}\setcounter{magicrownumbers}{0}
\begin{tabular}{|c|l|l|}
	\hline
	 \thead{№}   &  \thead{Обов’язкова умова допуску до \control у}  & \thead{Критерій} \\\hline
	 \rownumber      &  Потчний рейтинговий бал                          & $\ge 40$         \\ \hline
	 \rownumber      &  МКР                                              & виконана         \\ \hline
	 \rownumber      &  РГР                                              & здана            \\ \hline
\end{tabular}%
\end{center}

Додаткові умови допуску до \control у, які заохочуються:
\begin{enumerate}[label=$\bullet$]
    \item Залучення при виконанні РГР нових програмних засобів та застосунків для візуалізації результатів обрахунків, оптимізації обрахунків, використання оригінальних методик (додаються заохочувальні бали).
    \item Активна самостійна робота над теоретичним матеріалом: пошук та використання інформаційних ресурсів, ілюстрацій, відео, медіа ресурсів, що доповнюють поточний курс (додаються заохочувальні бали).
    \item Позитивний результат першої та другої атестації.
\end{enumerate}


\subsection*{Семестровий контроль (\control)}

\pgfmathsetmacro{\controlBalp}{15}
\pgfmathsetmacro{\controlBalu}{25}
\pgfmathsetmacro{\controlBal}{\controlBalp + \controlBalu}

\Control{} приймається у 2 етапи і складається із двох частин. Перша частина (контрольна робота)~--- виконується в екзаменаційну сесію напередодні другого етапу і має тривалість $3$ астрономічні години. Друга частина (усна частина)~--- усна відповідь за білетом (співбесіда), що містить два питання з теорії і проходить в окремий день (наступний або через день після проведення КР).

Контрольна робота передбачає роз'язок $4$-х задач. Кількість балів за кожну задачу та відповідність набраних балів оцінці в університетській шкалі встановлюється викладачами в білетах до контрольної роботи, що готують і проводять її. Максимальний рейтинговий бал за контрольну роботу~\controlBalp.

Загальна оцінка за \control{} складається із стартового рейтингу, отриманого протягом семестру, та рейтингових балів набраних під час \control у. Рейтингові бали (максимум \controlBalu) за усний \control{} нараховуються згідно наступних критеріїв:
\begin{enumerate}[label=$\bullet$]
\item від 20 до 25 --- повна правильна відповідь, 95\% інформації, наведено рисунки, позначення, є письмовий коментар щодо базових понять та законів, формулювання та терміни точні, терміни роз’яснено, повна правильна відповідь на уточнюючі запитання
\item від 15 до 20 --- правильна відповідь, 80\% інформації, наведено рисунки, позначення, є письмові коментарі щодо базових понять та законів, формулювання та терміни по суті правильні але не повні, терміни роз’яснено, правильна відповідь на уточнюючі запитання
\item від 10 до 15 --- по суті правильна але неповна відповідь, 70\% інформації, наведено рисунки та позначення, відсутні письмові коментарі щодо базових понять та законів, формулювання та терміни по суті правильні але не повні, терміни не роз’яснено, правильна відповідь на більшість уточнюючих запитання
\item від 5 до 10 --- відповідь неповна, 50\% інформації, не наведено рисунки та позначення, відсутні письмові коментарі щодо базових понять та законів, формулювання та терміни в основному правильні але не повні, терміни не роз’яснено, відповіді на уточнюючі запитання не повні
\item від 0 до 5 --- відповідь неповна, 30\% інформації, не наведено рисунки та позначення, відсутні письмові коментарі щодо базових понять та законів, формулювання та терміни в основному не повні, терміни не роз’яснено, відповіді на уточнюючі запитання не повні або відсутні
\end{enumerate}


Остаточна оцінка $\mathbf{RD}$ є сумою рейтингових балів отриманих за поточний контроль та балів отриманих на \control і після співбесіди зі студентом.

\begin{center}\setcounter{magicrownumbers}{0}
    \begin{spreadtab}{{tabular}{|c|l|c|c||c|}}
    \hline
    @ \thead{№}   & @ \thead{Контрольний захід}                & @ \thead{Бал}  & @ \thead{Кількість}  & @ \thead{Всього}   \\ \hline
    @ \rownumber               & @ Модульна контрольна робота         & \mkrBal        & 1                    & c2*d2           \\ \hline
    @ \rownumber               & @ РГР                                & \rgrBal        & 1                    & c3*d3           \\ \hline
    @ \rownumber               & @ Практичні заняття                  & \pracBal       & 1                    & c4*d4           \\ \hline
    @ \rownumber               & @ Лабораторні роботи                 & \labBal        & 1                    & c5*d5           \\ \hline
    @ \rownumber               & @ \firstcap\control                  & \controlBal    & 1                    & c6*d6           \\ \hline\hline
    @                          & @ \multicolumn{3}{l||}{Всього}                                               & sum(e1:[0,-1])  \\  \hline
    \end{spreadtab}
\end{center}


%\pgfmathsetmacro{\lectRSO}{10}
%\pgfmathsetmacro{\mkrRSO}{20}
%\pgfmathsetmacro{\refRSO}{30}
%\pgfmathsetmacro{\ekzRSO}{40}
%\begin{enumerate}
%	\item Робота протягом семестру:
%	      \begin{enumerate}[label=\alph*)]
%	      	\item Відвідування всіх лекцій, активність на лекціях. Максимальна кількість балів~--- \lectRSO.
%	      	\item МКР. Максимальна кількість балів~--- \mkrRSO.
%	      	\item Індивідуальне завдання. Максимальна кількість балів~--- \refRSO.
%	      \end{enumerate}
%	\item Екзамен:
%	      максимальна кількість балів~--- \ekzRSO.
%	      \begin{enumerate}[label=\alph*)]
%	      	\item Вичерпна відповідь $35-40$ балів;
%	      	\item Відповідь з незначними неточностями $25-35$ балів;
%	      	\item Правильні формулювання з відсутністю доведень		$10-25$ балів;
%	      	\item Грубі помилки при формулюванні з відсутністю доведень $5-10$ балів;
%	      	\item Незнання обов'язкових формул та співвідношень чи постановки основних задач $0$ балів.
%	      \end{enumerate}
%	\item Максимальний сумарний рейтинг складає:\pgfmathsetmacro{\RDRSO}{\lectRSO + \mkrRSO + \refRSO + \ekzRSO}
%	      \[\mathbf{RD} = \lectRSO + \mkrRSO + \refRSO + \ekzRSO  = \RDRSO \text{ балів}.\]
%\end{enumerate}
%
%Календарний контроль: проводиться двічі на семестр як моніторинг поточного стану виконання вимог силабусу.


Таблиця відповідності рейтингових балів оцінкам за університетською шкалою.
%\footnote{Оцінювання результатів навчання здійснюється за рейтинговою системою оцінювання відповідно до рекомендацій Методичної ради КПІ ім. Ігоря Сікорського, ухвалених протоколом №7 від 29.03.2018 року.}.

\begin{center}
\begin{tabular}{|c|c|c|}
	\hline
	 \thead{Значення рейтингу}                  &     \thead{Оцінка ECTS}                \\ \hline
	 $95 \le \mathbf{RD} \le 100$               &         відмінно                       \\ \hline
	 $85 \le \mathbf{RD} < 95$                  &         дуже добре                     \\ \hline
	 $75 \le \mathbf{RD} < 85$                  &           добре                        \\ \hline
	 $65 \le \mathbf{RD} < 75$                  &        задовільно                      \\ \hline
	 $60 \le \mathbf{RD} < 65$                  &        достатньо                       \\ \hline
	 $ \mathbf{RD} < 60$                        &         незадовільно                   \\ \hline
	 \makecell[c]{Є не зараховані лабораторні роботи \\
     або не зарахована розрахункова робота}     &        не допущено                     \\ \hline
\end{tabular}%
\end{center}


%\section{Додаткова інформація з дисципліни (освітнього компонента)}

%\begin{itemize}
%\item перелік питань, які виносяться на семестровий контроль (наприклад, як додаток до силабусу);
%\item  можливість зарахування сертифікатів проходження дистанційних чи онлайн курсів за відповідною тематикою;
%\item  інша інформація для студентів/аспірантів щодо особливостей опанування навчальної дисципліни.
%\end{itemize}



\phantom{none}
\thispagestyle{empty}
\vfill
\printrequisites
\end{document}



