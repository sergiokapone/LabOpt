% !TeX program = lualatex
% !TeX encoding = utf8
% !TeX spellcheck = uk_UA
% !TeX root =../LabWork.tex


\keywords{когерентність, інтерференція світла, оптична довжина ходу хвилі, інтерференційна схема}
\abstract{Ознайомлення з явищем інтерференції в тонких плівках (смуги рівної товщини) на прикладі кілець Ньютона і з методикою інтерференційних вимірювань кривизни скляної поверхні.}
\chapter{Кільця Ньютона}
\makeworktitle


\section{Теоретичне підґрунтя}

Кільця Ньютона утворюються при інтерференції світлових хвиль, відбитих від границь тонкого повітряного прошарку, який знаходиться між опуклою поверхнею лінзи і плоскою скляною пластинкою (рис.~\ref{pic1}). Спостереження ведеться у відбитому світлі.

\begin{wrapfigure}{r}{0.4\linewidth}
	\begin{center}
		\begin{tikzpicture}[decoration={ markings,  mark=at position 0.5 with {\arrow{stealth}}},]
%    	\draw (-3,-3) to [grid with coordinates] (3,3);
        \fill[cyan, top color = cyan!50,  bottom color = cyan!50, middle color = white, draw=cyan] (-3,-0.25) rectangle (3,0.25);
%        \draw[dashed] (0,7.5) circle (7);
        \fill[cyan, top color = cyan!50,  bottom color = cyan!50, middle color = white, draw=cyan] (0,7.5) ++(244:7) arc (244:296:7) -- cycle;
        \draw (0,7.5) -- node[right] {$R$}++(287:7);
        \draw (0,7.5) -- ++(270:8.3);
        \draw [stealth reversed-stealth reversed] (2.6,0.9) -- node [right] {$d$} (2.6,0.4);
        \draw [stealth reversed-stealth reversed] (2,0.6)    --  node [left] {$d_0$} (2,0.15);
        \draw [thin, dashed] (-3,0.8) -- (3,0.8);
        \draw [thin, dashed] (0,0.50) -- (3,0.50);
        \draw (-2,0.8) -- ++(0,-1.6);
        \draw [latex-latex] (-2,-0.6) -- node[below] {$r_m$} (0,-0.6);
        \draw [red, postaction={decorate}] (-2.8, 4) node[left, black] {$1$} -- (-2.8,1.08) ;
        \draw [red, -stealth] (-2.8,1.08) -- (-2.2,0.25) -- (-2,0.8) -- (-1.2,1.6) node[above right, black] {$2'$} node[above left, black] {$1'$};
%        \draw[-stealth, red] (-2.8,1.08) -- (-2.5, 1.6);
        \draw [red, postaction={decorate}] (-2, 4) node[right, black] {$2$} -- (-2,0.8);
	\end{tikzpicture}
	\end{center}
	\caption{Утворення кілець Ньютона}
	\label{pic1}
\end{wrapfigure}
Нехай на систему згори падає монохроматичний паралельний пучок променів. Частина променів (промінь 1 на рис.~\ref{pic1} ) відбивається від верхнього краю пластини, а інша частина(промінь 2 на рис.~\ref{pic1}) від нижнього краю лінзи.

Промені $1'$ та $2'$ когерентні,  але між ними виникає різниця ходу. Роль тонкої плівки виконує повітряний проміжок між пластиною та лінзою.Нехай на систему згори  падає монохроматичний паралельний пучок променів. Частина променів (промінь 1 на рис.~\ref{pic1}) відбивається від верхнього краю пластини, а інша частина(промінь $2$ на рис.~\ref{pic1}) від нижнього краю лінзи. Промені $1'$ та $2'$ когерентні,  але між ними виникає різниця ходу. В першому наближені, як що знехтувати невеликим нахилом променів у повітряному зазорі, геометрична різниця ходу дорівнює:
\begin{equation}
\delta' =2(d+d_{0})
\end{equation}
де $d_{0}$~--- товщина зазору в місці контакту лінзи та пластини, яка може бути як додатною, наприклад, за наявності часток пилу між лінзою та пластиною, який викликає деформацію; $d_{0}+d$~--- товщина повітряного зазору на відстані $r_m$ від центру лінзи. Для того, щоб визначити повну  різницю ходу $d$  треба прийняти до уваги зміну фаз світлової хвилі під час відбиття від гранці поділу скло-повітря,коли показник заломлення першого середовища більше за показник заломлення другого, та під час відбиття від гранці повітря-скло, коли навпаки показник заломлення першого середовища менше за показник заломлення другого. Відомо, що для електричного вектора у першому випадку відбиття відбувається  без зміни фаз, а в другому призводить до зміни  фаз на $\pi$; фаза магнітного вектора, навпаки,змінюється на $\pi$ тільки під час першого відбиття. Таким чином, промені 1 і 2 набувають різниці фаз $\pi$, що відповідає додатковій різниці ходу $\lambda/2$, а повна різниця ходу:

\begin{equation}
\delta =2(d+d_{0})+\lambda/2.
\end{equation}

Якщо форма лінзи близька до сферичної з радіусом кривизни $R \gg r_m$, то геометричного міркувань $r^2_m = 2Rd $, і 
\begin{equation}
\delta=r^2_m/R+2d_{0}+\lambda/2.
\end{equation}

Якщо повна різниця ходу дорівнює $\lambda(m+1/2)$, то промені 1 і 2 гаситимуть один одного і спостерігатимуться темні плями(кільця). Радіус цих кілець $r_\text{тем}$ легко розрахувати за формулою:
\begin{equation}
\label{rdark}
r_\text{тем}^2=R(\lambda m-2d_{0})
\end{equation}

Аналогічно, для радіуса світлих кілець $r_\text{світ}$ маємо:

\begin{equation}
\label{rlight}
r_\text{світ}^2=R(\lambda m-2d_{0}-\lambda/2)
\end{equation}

Отже, за графіком залежності $r^2_m$ від номеру кільця можна визначити радіус кривизни лінзи, а також величину проміжку в місці контакту.

\section{Експерементальні подробиці}

\begin{wraptable}{O}{0.5\linewidth}
	\centering
	\begin{tabular}{lc}
		\toprule
		Колір                          & Довжина хвилі $\lambda$,~нм \\ \midrule
		\rowcolor{violet!10}Фіолетовий &             $ 404 \pm 10$             \\
		\rowcolor{blue!10}Синій        &             $ 434 \pm 10$             \\
		\rowcolor{cyan!10}Блакитний    &             $ 486 \pm 10 $             \\
		\rowcolor{green!10}Зелений     &             $ 546 \pm 10 $             \\
		\rowcolor{yellow!10}Жовтий     &             $ 586 \pm 10 $             \\
		\rowcolor{orange!10}Оранжевий  &             $ 656 \pm 10 $             \\
		\rowcolor{red!10}Червоний      &             $ 706 \pm 10 $             \\ \bottomrule
	\end{tabular}
	\caption{Довжини хвиль}
	\label{WaveLength}
\end{wraptable}
В даній лабораторній роботі кільця Ньютона досліджується за допомогою мікроскопа. На столику мікроскопа розташоване держак, на якому розміщується досліджувана лінза з пластиною. В одному з окулярів мікроскопа встановлюється освітювач, що генерує пучок променів, паралельних тим, що падають в околі спостерігача. Для монохроматизації пучка перед освітлювачем встановлюють фільтр. В комплект входять 7 фільтрів, що створюють  монохроматичні пучки, довжини хвилі яких наведені в таблиці~\ref{WaveLength}.

На початку експерименту рекомендуються знайти кільця Ньютона в білому світлі (без фільтра) і сфокусувати мікроскоп під своє око. Перехрестя шкал мікроскопа повинно проходити через центр кілець. Після цього можна встановити фільтр і переходити до безпосередніх вимірювань радіусу кілець. Для вимірювань  на окулярі мікроскопа нанесено спеціальну шкалу з поділками. Ціну поділки для кожного значення збільшення вказано в інструкції до мікроскопа.

Вимірювати радіус кілець слід від центру системи до середини кільця. Для збільшення точності рекомендуємо після першої серії вимірів із заданим фільтром повернути лінзу на $90^\circ$ навколо вертикальної осі  і повторити виміри. Якщо робота виконується двома студентами, то рекомендуємо провести виміри кожному з студентів, а потім порівняти й усереднити одержані результати.


\section*{Контрольні запитання}
\addcontentsline{toc}{section}{Контрольні запитання}
\begin{enumerate}[label*=\arabic*.]
    \item Від чого залежить кількість спостережуваних кілець?
    \item Чому кільця, що спостерігаються мають райдужне забарвлення? 
    \item  Чому по мірі віддалення від центру кільця розташовуються ближче один до одного?
    \item Що станеться з кільцями Ньютона, якщо проміжок між лінзою і пластинкою заповнити рідиною?
    \item Чи можна при спостережені кілець Ньютона у відбитому світлі отримати в центрі не темне, а світле кільце? Якщо так, то
    сформулюйте умови, які для цього необхідні.
\end{enumerate}
