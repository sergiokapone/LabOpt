% !TeX program = lualatex
% !TeX encoding = utf8
% !TeX spellcheck = uk_UA
% !TeX root =../LabWork.tex

\part{Дифракція світла}

\nocite{akhmanov, Godzhaev}
\printbibliography[title={Рекомендована література}, heading=subbibliography]

\section{Означення}

Дифракція світла --- це сукупність фізичних явищ, обумовлених хвильовою природою світла і спостерігаються при його поширенні в середовищі з різко вираженою оптичної неоднорідністю (наприклад, при проходженні через отвори в екранах, поблизу меж непрозорих тіл тощо). У більш вузькому сенсі під дифракцією розуміють огинання світлом різних перешкод, тобто відхилення від законів геометричної оптики. Точна теорія дифракції навіть для простих випадків є собою досить складною в математичному відношенні. Серед багатьох методів наближеного розв'язання задачі історично першим, найбільш простим і наочним, був метод, який в даний час прийнято називати \emph{принципом Гюйгенса-Френеля}. Гюйгенс, вперше обґрунтував хвильову теорію світла, запропонував наступну побудова: кожна точка довільного хвильового фронту стає джерелом елементарних сферичних вторинних хвиль, при цьому хвильовий фронт в будь-який інший момент є огинаючою цих вторинних хвиль. Френель доповнив принцип Гюйгенса твердженням про когерентність джерел вторинних хвиль, що дозволило йому розглядати основні дифракційні явища як результат інтерференції вторинних хвиль. Це поєднання побудови Гюйгенса з принципом інтерференції Френеля отримало назва \emph{принципу Гюйгенса-Френеля}, який, хоча і є наближеним, дозволяє кількісно описати дифракційні явища, які спостерігаються на простих об'єктах.

Математичне обгрунтування принципу Гюйгенса-Френеля було в подальшому дано Кірхгофом, який, зокрема, показав, що в якості поверхні вторинних джерел може бути вибрана не тільки поверхня хвильового фронту, але і будь-яка замкнута поверхня, всередині якої знаходиться точка спостереження. Пояснимо далі ідею Кірхгофа.

\begin{wrapfigure}{O}{0.45\linewidth}
	\centering
	\begin{tikzpicture}
		\draw[fill=gray!50] (-0.9,-1.9) -- (-0.9,1.9) -- (0.9,2.5) -- (0.9,-1.2) -- cycle;
		\draw[fill=white] (0.02,0.3) ellipse (0.6 and 1.5) ;
        \node at (0,0) {$\Sigma$};
        \draw[*-latex, dash pattern=on 1.34cm off 0.49cm] (-2,0)  node[left] {$S_0$} -- node[pos=0.3,above] {$r$} ++(35:2.5) coordinate (A);
        \draw[dashed] (A) -- ++(35:2) coordinate[pos=0.5] (A1);
        \draw[-latex] (A) --  node[above] {$s$} (4,0) coordinate (P) node[right] {$P$};
        \draw (A1) arc(35:-20:1) node[pos=0.5,right] {$\alpha$};
	\end{tikzpicture}
	\caption{Ілюстрація}
	\label{pic:Diffraction}
\end{wrapfigure}
Нехай на шляху сферичної монохроматичної світлової хвилі, що виходить із точкового джерела $S_0$, знаходиться плоский непрозорий об'єкт з отвором $\Sigma$, розміри якого великі в порівнянні з довжиною хвилі (рис.~\ref{pic:Diffraction}). Відповідно до принципу Гюйгенса-Френеля напруженість поля в точці $P$ за об'єктом визначається суперпозицією хвиль від вторинних джерел, розташованих в площині отвору $\Sigma$. При цьому амплітуда і фаза вторинних
сферичних хвиль, що приходять в точку $P$, залежать як від відстані $r$ (від джерела $S_0$ до відповідних ділянок хвильового фронту, що лежить на поверхні $\Sigma$), так і від відстані $s$ (від цих ділянок до точки $P$).

У загальному випадку комплексна амплітуда поля $E_P$ може бути знайдена з
допомогою інтегральної формули Френеля-Кірхгофа:

\begin{equation}\label{eq:Fr-Kirch}
    E_P = \iint\limits_{\Sigma}  \frac{A e^{ikr}}{r}K(\alpha) \frac{e^{iks}}{s} dS,
\end{equation}
де $k = \frac{2\pi}{\lambda}$~--- хвильове число, $\alpha$~--- кут між напрямками $r$ та $s$, $K(\alpha)$~--- фактор Френеля, що описує залежність амплітуди вторинних хвиль від кута $\alpha$ між напрямками поширення падаючої і вторинної хвилі, $dS$~--- елемент площі в площині отвору $\Sigma$,  $A$~--- константа. Інтегрування ведеться по <<відкритій>> в об'єкті поверхні $\Sigma$.

У цій формулі множник $\frac{e^{ikr}}{r}$ описує сферичну хвилю, що поширюється від точки $S_0$ до довільного вторинного джерела, розташованого на поверхні $\Sigma$, множник $\frac{e^{iks}}{s}$~--- сферичну хвилю, що йде від вторинного джерела до точки спостереження $P$.

Найбільш цікавим для розгляду є випадок, коли характерний лінійний розмір отвору малий у порівнянні з відстанями $r$ і $s$ від точок $S_0$ і $P$ до об'єкта. У цьому випадку як фактор $K(\alpha)$, так і множник $\frac{1}{rs}$ незначно змінюються при інтегруванні по отвору $\Sigma$ і основну роль в обчисленні дифракційної картини за формулою~\eqref{eq:Fr-Kirch} відіграє інтеграл від швидко множника вигляду $e^{ik(r+s)}$, який швидко осцилює. Розкладання в ряд цього множника дозволяє істотно спростити формулу~\eqref{eq:Fr-Kirch}. Явища, які описуються в рамках такого наближення, носять назва дифракції Френеля, або дифракції в ближній зоні. При $r \to \infty$ фронт падаючої хвилі можна вважати плоским. Якщо $s \to \infty$, то і вторинні хвилі, поширюються під деяким кутом $\alpha$ до початкового напрямку, утворюють плоский хвильовий фронт. Дифракційні явища, які спостерігаються при цих умовах, носять назву дифракції Фраунгофера, або дифракції в далекій зоні. Кількісний критерій, що дозволяє розрізняти наближення Френеля і Фраунгофера, буде наведено нижче після введення поняття зон Френеля.

\begin{figure}[h!]
	\centering
\begin{tikzpicture}[rotate=-90]
    \def\R{4} % sphere radius
    \def\Elevation{25} % elevation angle
    \fill[ball color=white!10] (0,0) circle (\R); % 3D lighting effect
    \foreach \i in {30,40,...,89} {\DrawLatitudeCircle[black]{\i}}
    \draw (0,\R) -- node[below] {$b$} (0,12) coordinate (P) node[below] {$P$};
    \draw (0,0) node[below] {$S_0$} -- node[below] {$a$} (0,\R);
    \NewLatitudePlane[Frensel]{\R}{15}{-180-30};
    \path[Frensel] (0:\R) coordinate (Pprime);
    \draw[] (0,0) ++(0,0.5)arc (90:{180-30}:0.5) node[pos=0.3, anchor = south west] {$\theta$};
    \draw (0,0) -- node[pos=0.6,anchor = east] {$a$} (Pprime);
    \draw (Pprime) -- node [above, sloped, rotate=-90] {$s=b+m\frac\lambda2$} (P);
    \fill[red] (0,0) circle (0.08);
    \fill[red] (P) circle (0.08);
\end{tikzpicture} 
	\caption{Ілюстрація методу зон Френеля}
	\label{pic:Diffraction2}
\end{figure}

Розглянемо надалі круглий отвір на який падає сферичний фронт хвилі. Оскільки інтегрування буде вестись по поверхні фронту, що увійшов в отвір, то $r = a$ змінюватись при інтегруванні не буде. Крім того, елемент поверхні фронту матиме вигляд:
\[
    dS = 2\pi a^2 \sin\theta d\theta = 2\pi \frac{a}{a + b} s ds
\]
 а тому, інтеграл~\eqref{eq:Fr-Kirch} матиме вигляд:
\begin{equation}\label{eq:Fr-KirchSpher}
    E_P =  \frac{2\pi Ae^{ika}}{a + b} \int K(\alpha) e^{iks} ds.
\end{equation}

Точне обчислення інтегралу~\eqref{eq:Fr-KirchSpher}, звичайно, неможливо без знання виду функції $K(\alpha)$. Однак Френель, використовуючи той факт, що довжини світлової хвилі, дуже мала, дав метод наближеного обчислення подібних інтегралів, який називається \emph{методом зон Френеля}. Розіб'ємо сферичний фронт на кільцеві області таким чином, щоб відстань від границь цих областей до точки спостереження дорівнювала $b$, $b + \frac\lambda2$, \ldots,  $b + m\frac\lambda2$, \ldots  (рис.~\ref{pic:Diffraction2}). Ці кільцеві області називаються \emph{зонами Френеля}. Зважаючи на малість довжини хвилі, функція $K(\alpha)$ в межах однієї зони може вважатися постійною. У цьому наближенні інтеграл~\eqref{eq:Fr-KirchSpher} по $m$-й зоні буде дорівнювати:
\begin{equation}\label{eq:Fr-KirchSpher_m}
    E_m = \frac{2\pi Ae^{ika}}{a + b} \cdot K_m\int\limits_{r + (m-1)\frac\lambda2}^{r + m\frac\lambda2}   e^{iks} ds = \frac{2\pi Ae^{ika}}{a + b} \cdot (-1)^{m+1}K_m \frac{e^{ikb}}{ik},
\end{equation}
де $K_m$~--- фактор Френеля для $m$-ї зони.

Результат дії усіх зон в точці $P$ є сумою амплітуд усіх зон:
\begin{equation}\label{eq:SumAmpl}
    E_p = \sum\limits_{m} E_m =\frac{2\pi \frac{A}{a+b}e^{ik(a + b)}}{ik}  \sum\limits_{m} (-1)^{m+1}K_m
\end{equation}

Те, що знаки сусідніх  доданків в сумі~\eqref{eq:SumAmpl} протилежні, означає, що коливання, що вносяться сусідніми зонами Френеля, протилежні по фазі, це слід було очікувати, оскільки вже із самої побудови зон Френеля видно, що  коливання  сусідніх зон запізнюються одне відносно одного на половину довжини хвилі.

Отже, якщо на шляху хвильового фронту поставити перешкоду у вигляді отвору, наприклад, який відкриває $N$ зон Френеля, то неважко показати, що результуюча амплітуда в точці $P$ буде дорівнювати:
\begin{equation}
    E = 
    \begin{cases}
        E_1 + |E_N|, \quad N = 2m \\
        E_1 - |E_N|, \quad N = 2m+1,
    \end{cases}
\end{equation}
Тобто, якщо отвір відкриває парне число зон $N = 2m$, то в точці $P$ спостерігатиметься максимум (світда пляма), а якщо отвір відкриває непарне число зон  $N = 2m + 1$ , то в точці $P$ спостерігатиметься мінімум (темна пляма), а навколо цієї точки чергуватимуться темні та світлі кільця. Однак, оскільки $|E_1|>|E_2|> \ldots >|E_N|$, то у випадку, якщо отвір відкриває багато зон Френеля $N \gg 1$, то $|E_1| \gg |E_N| $, тому амплітуда $E \approx \frac12 E_1$, що означатиме що в центрі буде завжди світла пляма.

Таким чином за умов дифракції на круглому отворі діаметру $D$ від точкового джерела, інтенсивіність світла обумовлюється кількістю відкритих зон $N$ пов'язаних таким співвідношенням:
\begin{equation}\label{eq:Zone_Diameter}
    \frac{D}{2}=r_{m}=\sqrt{m\lambda\frac{ab}{a+b}},
\end{equation}
де $r_{m}$~--- радіус $m$-ї зони Френеля. 

 Оскільки сумарна амплітуда~\eqref{eq:SumAmpl} є знакозмінним рядом, то закривши парні, або непарні зони, можна значно підвищити амплітуду в точці $P$. Для цього треба виготовити екран, який для деяких конкретних значень $a$ та $b$ відкривав би тільки парні або непарні зони. Тоді хвилі від відкритих зон надходили б у точку $Р$ синфазно і інтерференційно підсилювали одну одну. Такий екран називають \emph{зонною платівкою Френеля}. Переписавши формулу \eqref{eq:Zone_Diameter} в вигляді
\begin{equation}\label{eq:Focuse}
    \frac1b+\frac1a=\frac{m\lambda}{r_{m}^{2}}
\end{equation}
    отримаємо формулу для радіусів кілець зонної платівки Френеля. Порівнявши \eqref{eq:Focuse} з формолою тонкої лінзи $\frac1b + \frac1a  = \frac1f$, можна зробити висновок, що зонна платівка працює як лінза з фокусною відстанню
\begin{equation}
    f=\frac{r^{2}_{m}}{m\lambda}.
\end{equation}

Для того, щоб визначити характер дифракції користуються критерієм. Оскільки, вигляд дифракційної  картини залежить від того скільки відкрито зон Френеля, то для визначення умов дифракції зручно ввести параметр, який є числом відкритих зон
\begin{eqnarray}
    m =\frac1D{\sqrt{\lambda\frac{ab}{a + b}}}
\end{eqnarray}
    де $D$~--- характерні розміри отвору.  За умови $m\gg1$ дифракційні ефекти незначні, і розподіл інтенсивності можна описати на основі  геометричної оптики. Для $m\approx1$, отвір або екран перекривають декілька зон Френеля, і має місце дифракція Френеля. Для $m\ll1$ відкрита лише незначна частина першої зони Френеля~--- можна вважати хвильовий фронт плоским, тобто має місце дифракція в  паралельних променях, або \emph{дифракція Фраунгофера}.


